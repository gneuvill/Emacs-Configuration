% Created 2010-12-05 dim. 00:06
\documentclass[11pt]{article}
\usepackage[utf8]{inputenc}
\usepackage[T1]{fontenc}
\usepackage{graphicx}
\usepackage{longtable}
\usepackage{float}
\usepackage{wrapfig}
\usepackage{soul}
\usepackage{amssymb}
\usepackage{hyperref}


\title{NEWS}
\author{neuville}
\date{05 d��cembre 2010}

\begin{document}

\maketitle

\setcounter{tocdepth}{3}
\tableofcontents
\vspace*{1cm}
\section{Changes for ECB version 2.40.1}
\label{sec-1}


\subsection{Better compatibility with forthcoming Gnu Emacs 23.2}
\label{sec-1.1}


\subsubsection{Works out of the box with the Emacs-integrated CEDET-suite}
\label{sec-1.1.1}


    If Emacs >= 23.2 is used then CEDET is already integrated into Emacs and
    ECB can be used out of the box without further requirements.

    PLEASE NOTE: If ECB detects an author version of CEDET (as available at
    `\href{http://cedet.sourceforge.net'}{http://cedet.sourceforge.net'}) then ECB will ALWAYS try to use that one
    even if you use Emacs >= 23.2! This is for users who want to use latest
    Emacs >= 23.2 but still prefer using the latest author version of CEDET
    instead of the Emacs-integrated one.

    So if you want to use the Emacs-integrated CEDET-suite you have to ensure
    that no author-version of CEDET is in the `load-path'! This means that the
    library cedet.el of the author-version MUST NOT be loaded into Emacs (as
    described in the file INSTALL of CEDET)!
    This is a valid check: (locate-library ``semantic-ctxt'') must return nil!

    Details about the Requirements of ECB are available in the README.
    
\subsection{The special symboldef ECB-window is now stable and ready for usage}
\label{sec-1.2}


\subsubsection{\textbf{TODO} description what is new}
\label{sec-1.2.1}


\subsection{ECB now uses the partial reparse feature of semantic}
\label{sec-1.3}


   The main effect a user will notice is that edit actions which trigger just a
   partial reparse will just update the related node in the methods-window and
   not the whole methods-window. One of the main advantages is that this
   preserves the expand/collapse state of the methods-window.

   Editing examples which trigger only a partial reparse:
\begin{itemize}
\item editing a function- or methodbody
\item changing the name of a variable or function/method
\item modifying the return-type of a function/method
\item and some other\ldots{}
\end{itemize}
   In these cases only the related node in the methods-window will change
   nothing else, ie. the expand/collapse-states of all other nodes in the
   methods-window will not be changed.

\subsection{ECB eliminates duplicates in the directories-buffer}
\label{sec-1.4}


   There can be duplicates either \ldots{}. TODO description
   
\subsection{New default-value for `ecb-auto-update-methods-after-save': nil}
\label{sec-1.5}


   Switching off this by default makes sense because now ECB uses the
   idle-scheduler of semantic (if it is switched on but this is at least very
   recommended) and therefore the methods-buffer will be up-to-date most of
   the time. Therefore updating it after saving makes not much sense. In
   addition this auto update is always a full-fetch for the tags which does
   not preserve the expand/collapse-state of the methods-window whereas a
   partial reparse does (s.a.).
   
\subsection{The contents of the methods-window are updated even if currently hidden}
\label{sec-1.6}


   If at least the current ecb-window-layout contains a methods-window then
   the contents will be autom-updated even if the methods-window is currently
   invisible (e.g. because the ecb-windows are all hidden or another
   ecb-window is currently maximized. But the updates takes only place if the
   semantic idle scheduler is active (see instructions in the manuals of ECB
   and/or cedet).

   The advantage of this new behavior is that now editing changes done with
   invisible methods-window will be reflected immediately after showing the
   methods-window (e.g. by unhiding the ecb-windows via the command
   `ecb-toggle-ecb-windows' - bound to [C-c . l w]).

\subsection{New support for Bazaar as version-control system}
\label{sec-1.7}

   If Bazaar is supported by VC (means vc-bzr.el is distributed with Emacs)
   then ECB supports it out of the box.

\subsection{New default value of the option `ecb-history-make-buckets'}
\label{sec-1.8}


   This option offers now an additional choice `directory-with-source-path:
   This substituts the best (= deepest) matching source-path of the
   directory-window in the directory-bucket with the alias of this matching
   source-path (if an alias is set in the option `ecb-source-path'). A new
   face is used for this: `ecb-history-bucket-node-dir-soure-path-face'.

\subsection{`defecb-window-dedicator-to-ecb-buffer' replaces `defecb-window-dedicator'}
\label{sec-1.9}


   This is only relevant for people who program own special ecb-windows.

\subsection{Fixed Bugs}
\label{sec-1.10}


\subsubsection{Command `delete-windows-on' has only deleted one window}
\label{sec-1.10.1}

    Now this command works also with ECB as expected.

\subsubsection{Fixed a bug which prevents the ecb-windows from being vertically resizeable}
\label{sec-1.10.2}

    This bug occured only when using Gnu Emacs.

\subsubsection{Fixed a bug concerning horizontal mouse-scrolling of ecb-windows via}
\label{sec-1.10.3}

    modeline-click when the ecb-window was not the selected window.
    This  bug occured only when using Gnu Emacs.

\subsubsection{Fixed a bug which prevented `ecb-type-tag-display' from taking effect.}
\label{sec-1.10.4}


\subsubsection{Killing a buffer failed when history-window is not visible}
\label{sec-1.10.5}

    Killing a buffer failed in case the history buffer is not visible (e.g.
    when currently a layout without a history buffer is used or when another
    buffer is maximized and the history buffer thus hidden). This is fixed
    now.

\subsubsection{All compile-modes (e.g. compile, grep etc.) failed in XEmacs}
\label{sec-1.10.6}

    This came from a bug introduced in 2.40 but it is fixed now.

\section{Changes for ECB version 2.40}
\label{sec-2}


\subsection{ECB now requires full CEDET-suite being installed (at least version 1.0pre6)}
\label{sec-2.1}


   This means the support for the single packages semantic 1.4.X, eieio 0.X
   and speedbar 0.X is completely switched off.

   As a consequence ECB is no longer runnable as XEmacs-package via the
   package-manager of XEmacs. This is because CEDET is not available as
   XEmacs-package. Therefore ECB can also not be run as XEmacs-package. If
   CEDET will later become a XEmacs-package then probably ECB will come back
   as XEmacs-package. But in the meanwhile you have to install ECB ``by hand''.

   In general the current CEDET is much more powerful than older
   semantic-versions and ECB uses this power, especially the power of the
   semantic-analyzer.

\subsection{Completely reworked synchronizing mechanism of the ECB-windows}
\label{sec-2.2}


   Now there are separate options for synchronizing:

\begin{itemize}
\item The basic ECB-windows (directories, sources, methods, history):
     `ecb-basic-buffer-sync', `ecb-basic-buffer-sync-delay' and
     `ecb-basic-buffer-sync-hook'.
\item The analyser-window (s.a.): `ecb-analyse-buffer-sync',
     `ecb-analyse-buffer-sync-delay' and `ecb-analyse-buffer-sync-hook'
\item The symboldef-window: `ecb-symboldef-buffer-sync',
     `ecb-symboldef-buffer-sync-delay' and `ecb-symboldef-buffer-sync-hook'
\item The integrated speedbar (if used): `ecb-speedbar-buffer-sync',
     `ecb-speedbar-buffer-sync-delay' and `ecb-speedbar-buffer-sync-hook'
\item The integrated eshell-support (if used): `ecb-eshell-buffer-sync',
     `ecb-eshell-buffer-sync-delay' and `ecb-eshell-buffer-sync-hook'
\end{itemize}
   But all options for synchronizing not the basic ECB-windows offer an
   additional choice `basic: Then ECB uses the setting of the related option
   for the basic ECB-windows! Example: If you set `ecb-eshell-buffer-sync' to
   `basic then the value of `ecb-basic-buffer-sync' is used. If you set
   `ecb-eshell-buffer-sync-delay' to `basic then the value of
   `ecb-basic-buffer-sync-delay' is used.

   This enhancement allows much more control for the synchronizing of certain
   windows.
     
\subsection{Better compatibility with CEDET >= 1.0pre6 and semantic >= 2.0pre6}
\label{sec-2.3}


\subsubsection{The ECB-analyse-window now works very well with current semantic analyzer}
\label{sec-2.3.1}

    There is an own synchronize-option-set for the analyse-window:
    `ecb-analyse-buffer-sync', `ecb-analyse-buffer-sync-delay' and
    `ecb-analyse-buffer-sync-hook'. See the docstrings. The advantage is, that
    now the synchronizing of the analyse-window can be customized separately
    from the other ECB-windows. THis is important because for the
    analyse-window a higher delay (or even disabled auto.synchronizing)
    increases usability a lot. See the info-manual.
    
\subsubsection{Fix compatibility about some changes of semantic concerning adopting}
\label{sec-2.3.2}

    external member-functions (as in C++ or eieio).

\subsubsection{ECB now works very well with semantic-idle-scheduler-mode (s.a.)}
\label{sec-2.3.3}


\subsection{More user-responsible buffer-parsing based on the idle-mechanism of semantic}
\label{sec-2.4}


   Force a reparse of the semantic-source if the idle-scheduler is off.
   Generally ECB calls semantic to get the list of tags for current
   source-file of current edit-window. Per default ECB does never
   automatically force a reparse of the source-file - this is only done on
   demand by calling `ecb-rebuild-methods-buffer'. So per default the
   idle-scheduler of semantic is responsible for reparsing the source-file and
   when this is necessary (see `semantic-idle-scheduler-mode' for further
   details). This is the most user-resonsible and therefore the recommended
   approach.

   So it's strongly recommended to enable `semantic-idle-scheduler-mode'
   because then reparsing is always done during idle-time of Emacs and is also
   interruptable.

   But if this idle-scheduler is switched off then ECB offers now two
   possibilities (via `ecb-force-reparse-when-semantic-idle-scheduler-off'):
   
\begin{itemize}
\item Not forcing itself a reparse when tags are needed by ECB: then a user
     declines knowingly Emacs/semantic-driven parsing of code when he/she
     switches off the idle-mode of semantic. This is the default behavior of
     ECB and the default value of this option. But this has also the
     consequence that the methods-buffer is only filed on demand via
     `ecb-rebuild-methods-buffer' (bound to \[C-c . r])!

     This means also in consequence that the methods-buffer is not
     automatically filled when a source-file is opened but first on demand
     after calling the command `ecb-rebuild-methods-buffer'!
\item Forcing a reparse when tags are needed: Then ECB forces semantic to parse
     the source-file when ECB needs tags to display. For this behavior this
     option has to be set to not nil.
\end{itemize}
   The term "forcing a reparse by semantic" is a simplification: ECB uses then
   the function `semantic-fetch-tags' which can decide that the cached tags
   are up-to-date so no real reparsing is necessary - but it can also run a
   full reparse and this reparse is not being done when Emacs is idle but
   immediatelly and not interruptable (as with the idle-scheduler of
   semantic), which can be quite annoying with big source-files.

\subsection{Complete reworked history-buffer}
\label{sec-2.5}


\subsubsection{The history is able to deal with indirect-buffer entries.}
\label{sec-2.5.1}

    See new option `ecb-history-stick-indirect-buffers-to-basebuffer'.

\subsubsection{The history can now be bucketized, see new `ecb-history-make-buckets'.}
\label{sec-2.5.2}

    This option allows to define several criterias for building buckets in the
    history-buffer all the history entries are sorted in (e.g. by major-mode,
    directory, file-extension or regular expressions).

    You can change the bucketizing type on the fly via the popup-menu of the
    history-window.

    The regular expressions bucketizing allows in combination with the
    indirect-buffer ability to work with something like "virtual folders"
    (well, "virtual folders" light).

    For example, there is a large project with a huge number of files, and
    there are various tasks in this project. So it could be convenient to
    group buffers according to various tasks. It could be fulfiled through
    using indirect buffers, for example like this

       task_1-aaa.pl
       task_1-bbb.c
       task_1-ccc.sh
       ...
       task_N-xxx.java
       task_N-ccc.sh

    This means create indirect buffers with a name-part which can be used for
    grouping together buffers with same name-part (here e.g. task_1- ...
    task_N-). In the example above you would create two indirect buffers for
    the filebuffer ccc.sh, one named task_1-ccc.sh, the other named
    task_N-ccc.sh).

    Then use the new option `ecb-history-make-buckets' to define regexps for
    bucketizing all (indirect) buffers according their task-part in the
    buffer-name.

\subsubsection{There are now new faces for the history entries.}
\label{sec-2.5.3}

    See new options `ecb-history-bucket-node-face',
    `ecb-history-dead-buffer-face' and `ecb-history-indirect-buffer-face' and
    equaly named new faces.

\subsection{Sticky parent-node for all special ECB-windows of type tree-buffer}
\label{sec-2.6}


   Only for GNU Emacs: in the header-line of a tree-buffer always the
   current-parent node of the first visible node is displayed (if there is a
   parent node). This sticky node is exactly in the same manner clickable as
   all other nodes. There is a new option `ecb-tree-make-parent-node-sticky'
   which enabales/disables this new feature (default is enabled).

\subsection{Much saver advice-backbone for all advices needed by ECB}
\label{sec-2.7}


   This is not a user-visible change but enhances the stability of ECB by
   using now a new advice-backbone which guarantes that all ecb-advices are
   enabled rsp. disabled correctly depending on the surrounding context.
   Introducing four new macros `defecb-advice-set', `defecb-advice',
   `ecb-with-original-adviced-function-set' and `ecb-with-ecb-advice'.

\subsection{Changes in the version control support of ECB}
\label{sec-2.8}


\subsubsection{ECB now also offers version control support only if the current vc-package is}
\label{sec-2.8.1}

    installed which offers support for modern vc-backends like subversion, git
    and monotone.
    Because vc-package lacks having a version number which can be checked ECB
    uses the following check if the right vc-package is installed:
    (locate-library "vc-svn") must return not nil.
   
\subsubsection{New support for Git and Monotone as version-control systems}
\label{sec-2.8.2}

    If Git rsp. Monotone are supported by VC (means vc-git.el rsp. vc-mtn.el
    are distributed with Emacs) then ECB supports now both of them out of the
    box.

\subsubsection{`ecb-vc-supported-backends' offers now also a accurate recompute-state function}
\label{sec-2.8.3}

    This is by offering `ecb-vc-recompute-state' as an alias to the function
    `vc-recompute-state' as state-check function for this option.

\subsubsection{CVS-support is now fully compatible with Emacs 22 and 23}
\label{sec-2.8.4}


\subsection{New options rsp. commands}
\label{sec-2.9}


\subsubsection{New command `ecb-goto-window-edit-by-smart-selection'}
\label{sec-2.9.1}


\subsubsection{New command `ecb-goto-window-ecb-by-smart-selection'}
\label{sec-2.9.2}


\subsubsection{New option `ecb-ignore-pop-up-frames'}
\label{sec-2.9.3}

    The new option is for customizing the behavior of ECB concerning
    `pop-up-frames'. It allows three value: Always, only when a permanent
    compile-window is used and never. This makes ECB fully compatible with the
    option `pop-up-frames'.

\subsection{Much better compatibility with current Emacs-versions}
\label{sec-2.10}


\subsubsection{Full compatibility with Gnu Emacs 22}
\label{sec-2.10.1}


\begin{itemize}

\item `balance-windows' now works with Emacs 22 too\\
\label{sec-2.10.1.1}

     Cause of the completely new implemantation based on `window-tree' ECB
     uses a new machanism for enabling balance-windows to work properly with
     active ECB, so only the edit-windows are balanced but all ecb-windows
     remain on their sizes.


\item view-mode works also when a permanent compile-window is active\\
\label{sec-2.10.1.2}

     This is especially important for displaying help and completions.


\item `master-mode' now works with Emacs 22\\
\label{sec-2.10.1.3}



\item Grepping from ECB now uses per default `lgrep' rsp. `rgrep'.\\
\label{sec-2.10.1.4}


\begin{itemize}

\item The option `ecb-grep-function' defaults to `lgrep' if available\\
\label{sec-2.10.1.4.1}

      If not it tries `igrep' or 'grep'.


\item `ecb-grep-find-function' has been renamed to `ecb-grep-recursive-function'\\
\label{sec-2.10.1.4.2}

      It defaults to `rgrep' if available. If not it tries `igrep-find' or
      `grep-find'. The old value is automatically upgraded to the new option
      name.

\end{itemize} % ends low level

\item Fixed problems with `ecb-fix-window-size' and active compile-windows\\
\label{sec-2.10.1.5}

     With Emacs >= 22 the bugs of Emacs 21.3.X concerning `window-size-fixed'
     are fixed by the Emacs-team so now ECB supports this feature also with
     active compile-window.


\item Fixed small lack in the `switch-to-buffer-other-window'-advice.\\
\label{sec-2.10.1.6}

     In Emacs 22 this command ignores the settings in `same-window-*'. Now ECB
     adopts this behavior also for its adviced version so the command works
     in a smart manner optimized for ECB.

\end{itemize} % ends low level
\subsubsection{Much better compatibility with forthcoming Gnu Emacs 23}
\label{sec-2.10.2}


\begin{itemize}

\item Removed an annoying behavior which has set the mark hundred of times even\\
\label{sec-2.10.2.1}

     when the user just moves the cursor.
    
     This comes from a change within Emacs 23 and the function `goto-line'
     which sets the mark in Emacs 23. This is not necessary when not used as
     command but as internal utility. Now ECB uses its own goto-line-function
     which does not set the mark --> No unnecessary mark settings with Emacs
     23.


\item The adviced `display-buffer' is now fully compatible with Emacs 23\\
\label{sec-2.10.2.2}

     This means simply that you can work with ECB and Emacs 23 as you expect.

     Especially for displaying temporary buffers (like completion, help,
     compilation etc.) this means: Gnu Emacs 23 has a much smarter
     window-manager for displaying temporary buffers (like help-buffers) and
     "compile"-buffers. Therefore ECB uses completely the logic of Emacs 23
     and does exactly what Emacs 23 would do without activated ECB: This means
     mainly that the split-behavior depends on the settings of the options
     `split-width-threshold' and `split-height-threshold'.
      
     So if you are wondering why displaying temporary-buffers (like help,
     compile, grep, completion etc.) shows different behavior compared to
     deactivated ECB, then you probably have to adjust the values of the
     options `split-width-threshold' and `split-height-threshold' so they
     matches the sizes of the edit-area of ECB (at least nearly).


\item Fixing some incompatibilities with Emacs 23 concerning using temp. buffers\\
\label{sec-2.10.2.3}


     The fit-window-to-buffer version of Emacs 23 can destroy windows and is
     therefore not useable within ECB - now ECB has its own - stolen and
     adapted from Emacs 22.
  
     In addition to this the internal help-window-manager of Emacs 23 (a.o.
     responsible for displaying *Help*-buffers) chooses sometimes dedicated
     windows (which is a problem because ECB uses dedicated windows for its
     own special ECB-windows) whereas Emacs 22 doesn't do this. Now ECB works
     smoothly with Emacs 23 help-system. Same for view-mode.


\item ediff-support of ECB is now also fully compatible with Emacs 23\\
\label{sec-2.10.2.4}


     Emacs 23 uses dedicated windows: Now ECB can deal with frame-layouts where
     some windows are neither an ecb-, edit- nor a compile-window. Therefore
     now also single-frame mode (see `ediff-windows-setup-function') of ediff
     wirks smoothly with ECB.


\item All this together makes the new layout-engine of ECB much more robust so ECB\\
\label{sec-2.10.2.5}

     works smoothly with forthcoming Emacs 23.
    
\end{itemize} % ends low level
\subsubsection{Better compatibility with XEmacs 21.4 and XEmacs 21.5}
\label{sec-2.10.3}


    Compatibility has been increased but is still not perfect. There is a need
    for much more tests with XEmacs. This can not been done by the
    ECB-maintainer because it costs too much effort. Any help from
    XEmacs-gurus using ECB for their daily work is very appreciated.

\subsection{ECB is able to work with indirect buffers if the base-buffer is filebased}
\label{sec-2.11}


   Now you can work with indirect-buffers as well as with normal file-buffers,
   i.e. indirect buffers are shown in the history��, their contents are
   displayed in the methods-buffer, the ECB-analyse-buffer works with them,
   autom. synchronizing the ECB-tree-buffers works for them etc...

\subsection{Deprecated, disabled and deactivated features:}
\label{sec-2.12}

    
\subsubsection{The commands `ecb-download-ecb' and `ecb-download-semantic' are deactivated}
\label{sec-2.12.1}

    These features are not supported anymore. It costs two much effort to keep
    the needed links and mechanism to the sourceforge-download-area up to date.
    Please download the needed release-archives and extract and install as
    described in the shipped README rsp. INSTALL-files.

\subsection{Fixed Bugs}
\label{sec-2.13}


\subsubsection{Fixed a bug which displayed empty tag-names in the methods-buffer when font-lock}
\label{sec-2.13.1}

    is not loaded.

\subsubsection{Fixed a bug in the ping-functionality}
\label{sec-2.13.2}

    Now `ecb-ping-options' has a new default value which should operate well
    on most os.

\subsubsection{Fixed a bug which prevented `ecb-rebuild-methods-buffer' from working}
\label{sec-2.13.3}

    correctly

\subsubsection{Fixed a bug in the internals of the methods-filter which prevented}
\label{sec-2.13.4}

    all commands `ecb-methods-filter-*' from working correctly.

\subsubsection{Fixed context-menu for VC-operations in the sources- and history-window}
\label{sec-2.13.5}

    Now all vc-commands are called interactively by ECB.

\subsubsection{Fixed a bug with the ECB-navigation feature (C-c . p and C-c . n)}
\label{sec-2.13.6}

    Now the commands `ecb-nav-goto-next' and `ecb-nav-goto-previous' work
    saver and also for indirect buffers.

\subsubsection{Fixed small bugs with synchronizing current tag with highlighted tag}
\label{sec-2.13.7}

    in the methods buffer. This was when the buffer was reparsed with
    `semantic-idle-scheduler-mode' or after saving the buffer.

\subsubsection{Fixed a bug in `ecb-create-new-layout' which has displayed wrong modelines}
\label{sec-2.13.8}

    in the layout-creation frame.
    
\subsubsection{Fixed a bug in `ecb-submit-problem-report' which prevented it from working}
\label{sec-2.13.9}



\section{Changes for ECB version 2.32}
\label{sec-3}


\subsection{New backbone-add-on of ECB for getting the definition of the symbol at point}
\label{sec-3.1}


   It allows to display in a new special ECB-window the semantic (do not
   confuse this term with the semantic-library!) context of the definition of
   the current symbol under point. Per default ECB tries to find this context
   via semanticdb (part of cedet) and etags (shipped with (X)Emacs) but in
   general this is completely customizable; see the options of the new
   options-group "ecb-symboldef".

   Either use the layout "left-symboldef" (e.g. via [C-c . l c]) or create a
   new ecb-layout via the command `ecb-create-new-layout' and add a buffer of
   type "other" and name "symboldef" into this new layout.

   This add-on interactior can be seen as a "backbone" infrastructure, because
   it is/should be highly customizable how to get the definition of the symbol
   at point (e.g. via semanticdb or etags or whatever).

   Thanks to Hauke Wintjen <hauke.jans@tietoenator.com> for having the idea
   and writing the first implementation

   This add-on ECB-window of ECB is currently in beta or gamma stage but works
   already quite well. So please try it out and send feedback to the
   ECB-mailing-list. Every suggestion is highly appretiated!

\subsection{New native ECB-window (tree-buffer) for the semantic-analyzer}
\label{sec-3.2}

   There is a new tree-buffer ECB-analyse which displays the tags and
   informations of the semanic-analyzer in a smooth way. There are already new
   prebuild layouts "left-analyse" and "leftright-analyse" which contain this
   new tree-buffer ECB-window. But of course it can be added to own build
   layouts - either via the command `ecb-create-new-layout' (add a buffer of
   type "other" and name "analyse" into the new layout) or via a
   elisp-programmed new layout.
   
\subsection{New smart-arrow-navigation in a tree-buffer with up- and down-arrow}
\label{sec-3.3}

   Point jumps to the first character of the previous (up) rsp. next node
   (down). "First" character means either the first character of the
   expand-symbol (in case `ecb-tree-expand-symbol-before' is not nil) or of
   the displayed node-name. Or with other words: The first non-indentation and
   non-guide-line (see `ecb-tree-buffer-style') character of a node.

   Left- and right-arrow were already smart - see the existing option
   `ecb-tree-navigation-by-arrow'.
         
\subsection{Better handling of maximized ecb-windows}
\label{sec-3.4}


\subsubsection{Smarter handling of node-selections in maximized tree-windows}
\label{sec-3.4.1}


    Now the default behavior of ECB is:

    Maximized directories-window: When selecting a directory then first
    automatically the maximized directories-window will be "minimized" (i.e.
    all ecb-windows of current layout are displayed) if the current layout
    contains a sources-buffer and no sources are shown in the
    directories-window. So the source-files can be displayed in the
    sources-window.
    
    Maximized sources- or history-window: When selecting a source-file in one of
    these buffers then first automatically the maximized window will be
    "minimized" (i.e. all ecb-windows of current layout are displayed) if the
    current layout contains a methods-buffer. So the tag-contents of the
    selected source-file can be displayed in the methods-window.

    For a even smarter behavior ECB introduces a new option
    `ecb-maximize-next-after-maximized-select' which automatically maximizes
    the next logical tree-window after a node selection. The definition of
    "next logical is": Directories --> sources, sources/history --> methods.
    See the docstring of this new option for more detailled informations.
    
    Thanks for suggestion to Michael Reiher.
    
\subsubsection{Easier maximizing and "minimizing"}
\label{sec-3.4.2}

    ECB supports the default modeline-mechanisms for deleting other windows.
    GNU Emacs binds `mouse-2' in its modeline to `delete-other-window'. XEmacs
    binds a popup-menu with some window commands to `button-3' in its
    modeline.

    ECB combines the best of both worlds by supporting both of these
    mechanisms for both Xemacs and Emacs: ECB binds a toggle-command to
    `mouse-2' in the modeline of each tree-buffer which maximizes the current
    tree-window if all ECB-windows are visible and displays all ECB-windows if
    current tree-window is maximized. In addition ECB binds a popup-menu to
    `mouse-3' which offers exactly 2 commands: Maximizing current tree-window
    and displaying all ECB-windows.
   
\subsection{Renamed options or options with new type}
\label{sec-3.5}

   The following options have been renamed:
   ecb-truncate-lines --> ecb-tree-truncate-lines

   The following options have now a new type:
   ecb-tree-RET-selects-edit-window
   ecb-tree-image-icons-directories

   ECB automatically upgrades the old values to the new options!

\subsection{Better assistance fixing errors related to semantic-load problems}
\label{sec-3.6}

   Often users report errors somehow related to problems running ECB with the
   right semantic-version. There are different reasons for this:
   - People do not install the cedet-suite as described in the INSTALL-file of
     cedet.
   - People byte-compile ECB with another semantic (also eieio and speedbar)
     version than loaded into Emacs.
   All of them prevent ECB from working correctly.

   Now ECB detects the error at startup-time and assists the user in finding
   the reason of the problem and how to fix it.

\subsection{More robust downloading of new ECB from within Emacs}
\label{sec-3.7}

   Now `ecb-download-ecb' should work also with native Windows-XEmacs and also
   when ECB is installed in a directory which name contains spaces. If there
   are still problems in running this command please check the new options
   `ecb-wget-setup', `ecb-gzip-setup' and `ecb-tar-setup'.

\subsection{Support for (X)Emacs < 21 has been officialy removed.}
\label{sec-3.8}

   ECB requires now (X)Emacs >= 21. ECB stops activation with an appropriate
   error-message when (X)Emacs < 21 is detected.

\subsection{Full documentation of the library tree-buffer.el}
\label{sec-3.9}

   The ECB-info-manual contains a new section "tree-buffer" in the chapter
   "Entry points for elisp-programers". This describes in detail how to
   program with the library tree-buffer.el which is shipped with ECB.

\subsection{Fixed Bugs}
\label{sec-3.10}


\subsubsection{Version-controlled (VC) and not-VC history-entries are now sorted}
\label{sec-3.10.1}

    correctly according the setting in `ecb-history-sort-method'.

\subsubsection{Handles not anymore existing files in the stealthy tasks.}
\label{sec-3.10.2}

    If a stealthy task detects that a displayed file does not exist anymore it
    removes this source-file from the tree-displays.

\subsubsection{Fixed a bug when `ecb-process-non-semantic-files' is set to nil.}
\label{sec-3.10.3}

    This bug only occurs with XEmacs. As a side effect of this fix now the
    methods-buffer is correctly cleared out when loading a
    non-semantic-sourcefile (i.e. not parsable by semantic) and processing of
    non-semantic-files is switched off (`ecb-process-non-semantic-files' is
    set to nil) or the sourcefile is not parsable by a parser like imenu or
    etags/ctags.

\subsubsection{Fixed a bug in displaying semantic-tags which have plain-string members}
\label{sec-3.10.4}

    E.g. parsing `defstruct' has failed in previous ECB-version cause of this
    bug.

\subsubsection{Fixed a bug when Emacs is started with -q so Emacs could not determine the}
\label{sec-3.10.5}

    `custom-file'. This case is correctly handled now.



\section{Changes for ECB version 2.31}
\label{sec-4}


\subsection{Fixed bugs}
\label{sec-4.1}


\subsubsection{Fixed a recently introduced bug with XEmacs and an active isearch.}
\label{sec-4.1.1}

    This bug occured when a user has clicked onto a node in a tree-buffer
    during an active isearch and it had as result completely out-of-order
    tree-buffers. This buf is now fixed and should never occur.

\subsubsection{Fixed a bug concerning accessing root-drives in native window-XEmacs}
\label{sec-4.1.2}

    Fixes an annoying behavior of the native windows-version of XEmacs: When a
    path contains only a drive-letter and a : (e.g. C:) then
    `expand-file-name' (and a lot of other file-operations) do not interpret
    this path as root of that drive. So ECB adds temporary a trailing
    `directory-sep-char' and calls file-operations like `expand-file-name'
    with this new path because then `expand-file-name' treats this as root-dir
    of that drive. For all (X)Emacs-version besides the native-windows-XEmacs
    this fix is not needed and takes no effect.

\subsubsection{Better recognizing of remote CVS-root-repositories for the VC-support}
\label{sec-4.1.3}


\subsection{Enhancement for the VC-support}
\label{sec-4.2}


\subsubsection{New option `ecb-vc-xemacs-exclude-remote-cvs-repository'}
\label{sec-4.2.1}

    This excludes directories with a remote cvs-repository from VC-check. This
    option takes only effect for XEmacs and is needed cause of the outdated
    VC-package of XEmacs which offers no heuristic state-checking and also no
    option `vc-cvs-stay-local'. So this option takes only effect if
    `vc-cvs-stay-local' is not avaiable. In this case ECB treats directories
    which are managed by CVS but have a remote repository as if the directory
    would not being managed by CVS (so the files are not checked for their
    VC-state). This is done to avoid blocking XEmacs cause of running full
    cvs-commands (e.g. ``cvs status'') for a bunch of files over the net.

\subsubsection{Added beta-support for Clearcase}
\label{sec-4.2.2}

    There is already an identify-backend-function `ecb-vc-dir-managed-by-CC'
    and a check-state-function `ecb-vc-check-CC-state' and an advice for
    `clearcase-sync-from-disk' which enable automatic state-update after
    checkin/out. These code seems to work quite well but nevertheless there
    are some problems so this Clearcase-support is not added per default to
    `ecb-vc-supported-backends'.

    So if you want Clearcase-support you have to customize the option
    `ecb-vc-supported-backends' and add these functions mentioned above (the
    advice of `clearcase-sync-from-disk' takes only effect if the function
    `ecb-vc-dir-managed-by-CC' can be found in `ecb-vc-supported-backends' and
    ECB is active!). But use it at your own risk. Of course the best would be
    if a lot of people would test this Clearcase-support and fix the
    encountered problems ;-). Thanks to David Ostrovsky for providing a first
    implementation!



\section{Changes for ECB version 2.30.1}
\label{sec-5}


\subsection{Enhancement to the automatic option-upgrading mechanism}
\label{sec-5.1}

   ECB now automatically makes a backup-file of that file which will be
   modified by storing the upgraded rsp. renamed ECB-options. This backup file
   gets a unique name by adding a suffix ".before_ecb_<version>" to the name
   of the modified file. If such a file already exists ECB adds a unique number
   to the end of the filename to make the filename unique.
   This is a safety mechanism if something fails during storing the upgraded
   options, so you never lose the contents of your customization-file!
    
\subsection{Enhancement to the VC-support}
\label{sec-5.2}


\subsubsection{Better recomputing of the VC-state of a file when state changed outside}
\label{sec-5.2.1}

    With the new check-state-function `ecb-vc-state' the heuristic state is
    always computed right which is especially useful if the state for a file
    has been changed outside Emacs (e.g. by checking in from command line or
    Windows Explorer). This function is now added to the default-value of
    `ecb-vc-supported-backends' for GNU Emacs.

\subsubsection{Added out-of-the-box support for VC-system Subversion}
\label{sec-5.2.2}

    For this the latest version of the VC-package incl. the library vc-syn.el
    is needed. Latest CVS Emacs contains this VC-version. The new function
    `ecb-vc-dir-managed-by-SVN' is now added to the default-value of
    `ecb-vc-supported-backends'. Thanks for first implementation to Ekkehard
    G��rlach <ekkehard.goerlach@pharma.novartis.com>.

\subsection{Fixed bugs}
\label{sec-5.3}


\subsubsection{Fixed errors occured at load-time of ECB 2.30}
\label{sec-5.3.1}




\section{Changes for ECB version 2.30}
\label{sec-6}


\subsection{Enhancements to the file-browser}
\label{sec-6.1}


\subsubsection{Much better performance of the file-browser display because all}
\label{sec-6.1.1}

    time-consuming tasks (like the check if the displayed directories are
    empty of not) are now performed "stealthy" - means when Emacs is idle.
    Each stealthy task is interruptable by the user just by hitting any key or
    clicking the mouse so Emacs/ECB will not be blocked by such tasks; next
    time Emacs is idle again the interrupted task automatically proceeds from
    the state it has been interrupted. There is a new macro `defecb-stealthy'
    which can be used by a user to program own stealthy tasks.

    Currently ECB performs three stealthy tasks:

       Prescann directories for emptyness: Prescann directories and display
       them as empty or not-empty in the directories-buffer. See the
       documentation of the option `ecb-prescan-directories-for-emptyness' for
       a description.
     
       File is read only: Check if sourcefile-items of the directories- or
       sources-buffer are read-only or not. See documentation of the option
       `ecb-sources-perform-read-only-check'.
     
       Version-control-state: Checks the version-control-state of files in
       directories which are managed by a VC-backend. See the option
       `ecb-vc-enable-support'.

    There is also a new option `ecb-stealthy-tasks-delay'.

    There are three options which allow excluding certain directories from
    these stealthy tasks: `ecb-prescan-directories-exclude-regexps',
    `ecb-read-only-check-exclude-regexps' and last but not least
    `ecb-vc-directory-exclude-regexps'.

\subsubsection{ECB is now capable of handling remote paths.}
\label{sec-6.1.2}

    "Remote" means file- or directory-paths in the sense of TRAMP, ANGE-FTP or
    EFS. Such paths can now being added to the option `ecb-source-path' with
    no limitation compared to "local" paths. Just work with remote-paths in
    the same manner as with local paths. See also the additional choices of
    the options `ecb-prescan-directories-for-emptyness',
    `ecb-sources-perform-read-only-check' and `ecb-vc-enable-support' (new).
    This new support is tested with the combinations GNU Emacs+TRAMP and
    XEmacs+EFS but it should (hopefully) also work with all other
    combinations. Thanks a lot to Tomas Orti for beta-testing!

\subsubsection{ECB displays the Version-control-state of a file in the tree-buffers.}
\label{sec-6.1.3}

    There are four new options `ecb-vc-enable-support',
    `ecb-vc-supported-backends', `ecb-vc-directory-exclude-regexps' and
    `ecb-vc-state-mapping' which define if and how ECB should check the state
    of a sourcefile in a directory managed by a version-control system. By
    default ECB supports the same VC-backends as the builtin VC-support of
    Emacs: CVS, RCS and SCCS. But the option `ecb-vc-supported-backends'
    allows to add support for arbitrary VC-backends (e.g. Clearcase). New
    image-icons are also included for a cute display of the VC-state in the
    directories, sources and history-buffer. Thanks to Markus Gritsch
    <gritsch@iue.tuwien.ac.at> for contributing the icons.

    It's recommended to read the section "Version-control support" in the
    chapter "Tips and Trick" of the ECB-info-manual!

\subsubsection{New hook which runs directly after the selected directory has changed.}
\label{sec-6.1.4}

    See documentation of `ecb-after-directory-change-hook'.
    
\subsection{The popup-menu of the methods-browser allows precisely expanding of the}
\label{sec-6.2}

   current node. This means you can precisely expand a certain node to an
   exact indentation level relative to the node. This means all subnodes <=
   this level will be expanded (full recursive expanding is therefore of
   course also possible) and all subnodes indented deeper than this level will
   be collapsed - this is very different from using the expand/collapse symbol
   of a node. For forther details and examples the the manual and the section
   "Expanding" and here the subsection "Explicit expanding of the current node
   to a certain level".

\subsection{Automatically upgraded ecb-option-settings are now not saved by default.}
\label{sec-6.3}

   This means that ECB has now the new policy "Never touching the
   customization-files of a user without asking". The result is a completely
   redesigned upgraded-options-buffer: Now at the bottom of this buffer
   (displayed by `ecb-display-upgraded-options') two clickable buttons [Save]
   and [Cancel] are displayed which give the user the choice between saving
   the upgraded options for future Emacs-sessions ot just to cancel this
   buffer. In the latter case ECB has also upgraded the not compatible or
   renamed options (as listed in the displayed upgraded-options-buffer) but
   they will be not saved, i.e. no customization-file is touched and the
   changed and upgraded values will be lost after quiting Emacs.

\subsection{With XEmacs ECB temporary sets `progress-feedback-use-echo-area' to t}
\label{sec-6.4}

   This is necessary because otherwise the progress-display with native
   widgets modifies the window-sizes of ECB and does not exactly restore the
   window-sizes as before that progress-display. Deactivating ECB
   automatically restores the old value of this option.
   
\subsection{Fixed bugs}
\label{sec-6.5}


\subsubsection{Fixed resizing of the ecb-windows after opening a file}
\label{sec-6.5.1}

    Sometimes (X)Emacs (the behavior has only been reported for XEmacs)
    resizes the ecb-windows after opening a file by clicking onto a sourcefile
    or calling `find-file' (or similar functions). This is not a bug of ECB
    but nevertheless it is annoying for the ECB-users. Therefore ECB has now a
    workaround which prevents the ecb-windows from resizing. The work around
    is done via two simple advices of `find-file' and `find-file-other-window'.

\subsubsection{Fixed a bug in the upgrading feature (command `ecb-download-ecb') which has}
\label{sec-6.5.2}

    occured when the user has set a different LOKALE (e.g. "de_DE@euro").

\subsubsection{Fixed a bug in restoring sizes of the ecb-windows.}
\label{sec-6.5.3}

    Now a check will be performed if there are ecb-windows visible. If not
    nothing will be done (versions < 2.30 have failed in such a case).
    This bug has prevented ediff from working together with ECB when a
    compile-window was visible and the user has stored window-sizes for the
    current layout.



\section{Changes for ECB version 2.27}
\label{sec-7}


\subsection{The option `ecb-auto-expand-tag-tree-collapse-other' now has three possible}
\label{sec-7.1}

   choices. You can decide between auto-collapsing only when point stays onto
   a tag in the edit-window or even when point doesn't stay onto a tag (which
   then results in a full collapsed methods-buffer). ECB autom. upgrades the
   old setting of this option to the new type. Thanks to Javier Oviedo
   <joviedo@telogy.com> for the suggestion.

\subsection{Not accessible directories are displayed in a different face in the}
\label{sec-7.2}

   directory-browser. See the new option `ecb-directory-not-accessible-face'
   and the new face with same name.
   
\subsection{Enhancements to the permanent compile-window}
\label{sec-7.3}

   
\subsubsection{Enlarging the compile-window does not destroy some ecb-windows}
\label{sec-7.3.1}

    With previous versions of ECB enlarging the permanent compile-window has
    often destroyed some ecb-windows if the compile-window has occupied a
    large portion of the frame-height (e.g. half of the frame). Beginning with
    ECB 2.27 such a ecb-window-damage is either prevented or - if it still
    occurs (can not be prevented in all situations) - ECB now has a
    "self-healing"-mechanism which immediatelly repairs the layout via
    idle-checks if the current window-layout is correct for the current
    `ecb-layout-name'. Conclusion: Enlarging or shrinking the compile-window
    should never damage the ecb-windows! Thanks to Javier Oviedo
    <joviedo@telogy.com> for a great problem-report!

    Caution: If bytecompiled user-defined layouts are used (ie. the file,
    which contains the user-defined layouts is byte-compiled) then you MUST
    recompile this file with ECB 2.27 active!

\subsubsection{Now `ecb-store-window-sizes' and `ecb-restore-window-sizes' also work when}
\label{sec-7.3.2}

    called during a permanent compile-window is visible and also when these
    commands are called with another compile-window state than the other (e.g.
    when the window-sizes are stored without a compile-window and restored
    with a compile-window).

\subsubsection{Now ECB ensures that the permanent compile-window has always its specified}
\label{sec-7.3.3}

    size when shrinked either by Emacs/ECB or by the command
    `ecb-toggle-compile-window-height'.

\subsubsection{Now ECB ensures that all ecb-windows have always either their default or}
\label{sec-7.3.4}

    stored size size after the compile-window has been shrinked either by
    Emacs/ECB or by the command `ecb-toggle-compile-window-height'.

\subsubsection{ECB takes into account the new Emacs-option `grep-window-height' when it}
\label{sec-7.3.5}

    compute the max. allowed size of its permanent compile-window. More
    generally spoken: It takes into account all options with name
    *-window-height defined for modes which are defined by the macro
    `define-compilation-mode'. This mechanism is available in the forthcoming
    Emacs 21.4 (and of course with current CVS-Emacs 21).
    
\subsection{Fixed Bugs}
\label{sec-7.4}


\subsubsection{Added a workaround for a bug in the custom-library of current release 21.3}
\label{sec-7.4.1}

    of GNU Emacs which causes under certain circumstances the autom.
    option-upgrading-feature of ECB to fail. Same for the command
    `ecb-store-window-sizes'. Thanks for reporting this annoying bug and great
    help to Christoph Ludwig <cludwig@cdc.informatik.tu-darmstadt.de>



\section{Changes for ECB version 2.26}
\label{sec-8}


\subsection{Improved the performance of the directories-buffer-display}
\label{sec-8.1}

   Reduced the need of completely rebuilding the whole directories-tree-buffer
   when just switching between buffers and the related directories are already
   expanded. This can dramatically increase the speed of the tree-buffer
   display when displaying the sources-files in the directories-buffer (see
   `ecb-show-sources-in-directories-buffer'). Thanks to Rob Walker
   <rob.lists@tenfoot.org.uk> for tracking down the performance bottleneck and
   supplying a first patch.

\subsection{Per default all mouse-actions in the special ECB-buffers are now triggered}
\label{sec-8.2}

   first after releasing the mouse-button and not when pressed as with
   previous ECB-versions. This is the standard behavior of Emacs and of most
   applications. But with the new option `ecb-tree-mouse-action-trigger' a
   user can switch back to the old behavior.

   As a side effect now the ECB-tree-buffers can be used during an active
   isearch - at least with GNU Emacs. Thanks to Markus Gritsch
   <gritsch@iue.tuwien.ac.at> for pointing to this problem.

\subsection{ECB sets autom. temp-buffer-resize-mode rsp. temp-buffer-shrink-to-fit when}
\label{sec-8.3}

   needed. This modes are needed when a permanent compile-window is used
   because otherwise the correct sizing of the compile-window according to the
   settings in ecb-compile-window-temporally-enlarge and
   ecb-enlarged-compilation-window-max-height is not possible. After disabling
   the permanent compile-window or when deactivating ECB the original settings
   of these modes are restored!
      
\subsection{Some regexp-options has been changed to regexp-list-options:}
\label{sec-8.4}

   `ecb-excluded-directories-regexp' --> `ecb-excluded-directories-regexps'
   `ecb-source-file-regexps'         --> `ecb-source-file-regexps'
   `ecb-exclude-parents-regexp'      --> `ecb-exclude-parents-regexps'

   All these options now accept (and require) a list of regexps instead of
   exactly one big regexp. This allows easier regexp-definition and follows
   now the way Emacs goes with other regexp-options like `same-window-regexps'
   or `special-display-regexps' which are also lists of regexps.

   ECB autom. upgrades your old settings to the new option-types rsp. also
   -names (if the option has been renamed).

\subsection{New option `ecb-history-exclude-file-regexps' for excluding certain files}
\label{sec-8.5}

   from being displayed in the history-buffer of ECB. Currently the filenames
   TAGS and semantic.cache are excluded per default. Thanks to Javier Oviedo
   <joviedo@telogy.com> for the suggestion.

\subsection{ECB now displays read-only sourcefiles in a different face.}
\label{sec-8.6}

   For this a new option and a new face are offered both named
   `ecb-source-read-only-face' (default: italic) which is used to display
   source-files in the sources-tree-buffer (or in the directories-tree-buffer
   if `ecb-show-sources-in-directories-buffer' is not nil) if the file is
   read-only.
       
\subsection{Fixed bugs}
\label{sec-8.7}


\subsubsection{Fixed merging faces with XEmacs.}
\label{sec-8.7.1}

    Now options like `ecb-type-tag-display' (merges own ECB-faces to already
    by semantic faced tags) work also correct with XEmacs. ECB-versions prior
    to 2.26 has not merged the ECB-faces to the already applied faces of a
    text but instead replaced the applied faces with the ECB ones. Now all
    faces are merged.

\subsubsection{Now the command `ecb-expand-methods-nodes' and all the expand-* menu-entries}
\label{sec-8.7.2}

    in the popup-menu of the methods-buffer work for non-semantic-buffers too.

\subsubsection{Fixed a bug concerning `ecb-auto-expand-tag-tree-collapse-other'.}
\label{sec-8.7.3}

    Thanks to Javier Oviedo <joviedo@telogy.com> for pointing to the problem.

\subsubsection{Handles not existing source-paths in `ecb-source-path' correct.}
\label{sec-8.7.4}

    These paths are now ignored with a warning.

    

\section{Changes for ECB version 2.25}
\label{sec-9}


\subsection{More flexible sorting of the Sources- and the History-buffer}
\label{sec-9.1}


\subsubsection{`ecb-sort-history-items' has been renamed to `ecb-history-sort-method'}
\label{sec-9.1.1}

    and offers now the same choices as the already existing option
    `ecb-sources-sort-method'.

\subsubsection{Two new options `ecb-sources-sort-ignore-case' and}
\label{sec-9.1.2}

    `ecb-history-sort-ignore-case' which allow to ignore case when sorting the
    Sources- and/or the History-buffer. Thanks for suggestion to Markus
    Gritsch <gritsch@iue.tuwien.ac.at>. Per default case is now ignored.

\subsection{New icons for parent-display in the Methods-buffer.}
\label{sec-9.2}

   Thanks to Markus Gritsch <gritsch@iue.tuwien.ac.at> for contributing the
   icon-images.

\subsection{Fixed Bugs}
\label{sec-9.3}


\subsubsection{Fixed an annoying bug which results sometimes in an error "stack-overflow}
\label{sec-9.3.1}

    error in equal" when using `senator-try-expand-semantic' which is called
    when you use hippie-expand for example. Maybe the bug occured also in
    other situation but now this bug has been extirpated!

\subsubsection{Fixed a bug in the mechanism which prescanes directories for emptyness.}
\label{sec-9.3.2}

    Now a directory is checked if it is accessible before it is prescanned -
    otherwise the prescan could fail with an error.

\subsubsection{Fixed a bug in the autom. option-upgrading-mechanism.}
\label{sec-9.3.3}

    Thanks to Javier Oviedo <joviedo@telogy.com> for pointing to the
    underlying problem.



\section{Changes for ECB version 2.24}
\label{sec-10}


\subsection{Enhancements for the Methods-buffer}
\label{sec-10.1}

   
\subsubsection{Enhancements to the icon-display in the Methods-buffer}
\label{sec-10.1.1}


\begin{itemize}

\item New icon-images for enumerations and abstract/virtual tags (interfaces are\\
\label{sec-10.1.1.1}

     currently displayed with the icon for abstract classes - at least til
     someone provides special icons ;-)


\item Some slightly fixes in the other icons - e.g. a better display with a\\
\label{sec-10.1.1.2}

     not-white background


\item Static and abstract tags are no longer underlined rsp. italic when icons are\\
\label{sec-10.1.1.3}

     displayed because this is somehow "tautological" when a special static
     rsp. abstract icon is already displayed.

\end{itemize} % ends low level
\subsubsection{New option for switching on/off distinction of function-prototypes from the}
\label{sec-10.1.2}

    real function-definitions: `ecb-methods-separate-prototypes'. This is
    useful for C++ and C because these languages distinct between a
    method-prototype (rsp. function-prototype for C) and the method (rsp.
    function for C) itself. If this option is not nil (default) then ECB
    separates the prototypes from the real function/methods in the
    Methods-buffer.

\subsubsection{New "current-type"-filter for the Methods-buffer}
\label{sec-10.1.3}

    In languages which have types (like Java or C++) this filter displays only
    the current type and all its members \(e.g. attributes and methods). The
    current-type is either the type of the current tag in the source-buffer or
    the type of the current node in the methods-buffer. If ECB can not
    identify the current type in the source-buffer or in the methods-window
    then nothing will be done.
    
\subsection{Now directories are prescanned for emptyness so they are displayed as empty}
\label{sec-10.2}

   in the directories-buffer even without user-interaction (i.e. in previous
   ECB-versions the emptyness of a directory has been first checked when the
   user has clicked onto a directory). ECB optimizes this check as best as
   possible but if a directory contains a lot of subdirectories which contain
   in turn a lot of entries, then expanding such a directory or selecting it
   takes of course more time as without this check - at least at the first
   time (all following selects of a directory uses the cached information if
   its subdirectories are empty or not). Therefore this feature can be
   switched of via the new option `ecb-prescan-directories-for-emptyness'.

\subsection{Fixed Bugs}
\label{sec-10.3}


\subsubsection{Fixed some newly introduced bugs which made ECB 2.23 incompatible with}
\label{sec-10.3.1}

    semantic 1.4.

\subsubsection{Fixed a bug which changed the highlighted directory even after just expanding}
\label{sec-10.3.2}

    or collapsing other directories.

\subsubsection{Fixed two small bugs which have influenced `ecb-create-new-layout'}
\label{sec-10.3.3}


    

\section{Changes for ECB version 2.23}
\label{sec-11}


\subsection{New cedet1.0beta2 is supported.}
\label{sec-11.1}


\subsection{Enhancements for the Methods-buffer}
\label{sec-11.2}


\subsubsection{The look&feel of the Methods-buffer is now much nicer because it has been}
\label{sec-11.2.1}

    polished with a lot of new icons. ECB contains now a lot of new icons for
    displaying the Method-buffer as pretty and professional as possible. The
    new default-value of the option `ecb-tree-image-icons-directories' points
    autom. to the new icon-images. Thanks a lot to Markus Gritsch
    <gritsch@iue.tuwien.ac.at> for providing these new icons.

    To use this you should reset the option `ecb-tree-image-icons-directories'
    to the new default-value if you have customized this option!

    You can disable this icon-display via the new option
    `ecb-display-image-icons-for-semantic-tags' if you do not like the new
    icons. 

\subsubsection{New feature for applying default tag-filters for certain files.}
\label{sec-11.2.2}

    The new option `ecb-default-tag-filter' allow to define default
    tag-filters for certain files which are applied automatically after
    loading such a file into a buffer. The possible filters are the same as
    offered by the command `ecb-methods-filter' and they are applied in the
    same manner - the only difference is they are applied automatically. The
    files can be specified on a combined major-mode- and
    filename-regexp-basis.

    Usage-example: This can be used to display for outline-mode files (e.g.
    NEWS) in the Methods-buffer only the level-1-headings by defining a
    regexp-filter "^\* .*" (see the default-value of this new option).

\subsubsection{New keybindings and commands for faster applying certain tag-filters.}
\label{sec-11.2.3}

    Here are the new bindings:
    [C-c . fr] --> `ecb-methods-filter-regexp'
    [C-c . ft] --> `ecb-methods-filter-tagclass'
    [C-c . fp] --> `ecb-methods-filter-protection'
    [C-c . fn] --> `ecb-methods-filter-nofilter'
    [C-c . fl] --> `ecb-methods-filter-delete-last'
    [C-c . ff] --> `ecb-methods-filter-function'

\subsubsection{The popup-menu contains now mode-dependend tag-filter entries.}
\label{sec-11.2.4}

    This means for sources not supported by semantic no protection- or
    tag-class filters will be offered. And for semantic-supported sources
    exactly these tag-classes are offered the semantic-parser for the current
    major-mode offers. Same for the command `ecb-methods-filter'. For example
    texi-sources can only be filtered by the tag-classes "Definitions" and
    "Sections" and java-sources can be filtered by "Methods", "Variables",
    "Classes" etc. In general the semantic-variable
    `semantic-symbol->name-assoc-list' is used to get the right tag-classes.

\subsubsection{The option `ecb-show-tags' is now defined on a major-mode-basis.}
\label{sec-11.2.5}

    This means you can have different settings for each major-mode. ECB autom.
    upgrades your old setting to the new option-type.

\subsubsection{Distinction between functions and function-prototypes in the Methods-buffer}
\label{sec-11.2.6}

    This is for example useful for C++ and C because these languages distinct
    between a method-prototype (rsp. function-prototype for C) and the method
    (rsp. function for C) itself. The new default value of `ecb-show-tags'
    displayes per default the methods as flattened and the method-prototype as
    collapsed when opening a C- or C++-buffer. ECB autom. updates your old
    setting of this option to the new default value.

\subsection{The command `ecb-toggle-layout' now has a prefix-argument:}
\label{sec-11.3}

   If this optional argument is not nil (e.g. if called with a prefix-arg)
   then always the last selected layout was choosen regardless of the setting
   in `ecb-toggle-layout-sequence'. The last selected layout is always that
   layout which was current direct before the most recent layout-switch. So
   now a user can switch to another layout via `ecb-change-layout' [C-c . l c]
   and always come back to his previous layout via [C-u C-c . l t].

\subsection{Better internal self-monitoring of ECB}
\label{sec-11.4}

   ECB now monitors itself when Emacs is idle, ie. it monitors if all
   necessary ecb-functions are still member of the hooks `post-command-hook'
   and `pre-command-hook'. This is because Emacs resets these hooks to nil if
   any of the contained hook-functions fails with an error-signal. The
   ECB-hook-functions are error-save but if any of the "other" hook-functions
   of these hooks fail then the ECB-hooks would also have been removed from
   these hooks. The new monitoring-mechanism checks this periodically and
   re-adds the ECB-hooks if necessary.

\subsection{New option group for the integrated speedbar and one new option:}
\label{sec-11.5}

   The new hook `ecb-speedbar-before-activate-hook' runs directly before ECB
   activates the integrated speedbar. For example this hook can be used to
   change the expansion-mode of the integrated speedbar via
   `speedbar-change-initial-expansion-list'. Example:
   (speedbar-change-initial-expansion-list "buffers").

\subsection{Added a new section in the info-manual which describes how desktop.el can be}
\label{sec-11.6}

   used best in combination with ECB. This has been done because some
   conflicts have been reported. See the "Conflicts"-Chapter in the
   info-manual.
      
\subsection{Fixed bugs}
\label{sec-11.7}


\subsubsection{Double-clicking the mouse-1-button now works with integrated speedbar.}
\label{sec-11.7.1}


\subsubsection{Fixed two small bugs in the advice of `display-buffer':}
\label{sec-11.7.2}

    One bug occured when called for one of the special ECB-buffers. The other
    bug has not preserved the current-buffer when called for a buffer in the
    sense of `ecb-compilation-buffer-p' and a hidden compile-window is
    shown by display-buffer to display this buffer. The latter one for example
    has prevented the ant-server of JDEE from working properly.

\subsubsection{Fixed a bug so now always the correct directory in the directories-buffer}
\label{sec-11.7.3}

    is higlighted.

    

\section{Changes for ECB version 2.22}
\label{sec-12}


\subsection{New nifty feature for filtering the tags displayed in the Methods-buffer}
\label{sec-12.1}

   You can filter by regexp, protection, tag-class etc. See the new command
   `ecb-methods-filter' (bound to [C-c . f m] and the new option
   `ecb-methods-filter-replace-existing' for getting all details.

   Filters can be layered filters, i.e. you can apply filters on top of
   already applied filters: An example of 3 combined filters would be: Display
   only all public methods having the string "test" in its name. The modeline
   of the Methods-buffer always displays the topmost filter. But you can see
   all currently applied filters by moving the mouse over this section of the
   modeline: The filters will be displayed as help-echo.

   The new functionality is also completely available via the popup-menu of
   the Methods-buffer.

   Currently tags of class 'type can not be filtered out, because especially
   in OO-languages like Java or C++ this makes only less sense because if the
   class is filtered out then no filter can be applied to its members like
   methods or attributes.

   These tag-filters can also applied to sources which are not supported by
   the semantic-parser but "only" by imenu or etags. But because for these
   sources not all information are avaiable the protection- and tag-class
   filter can not used with such "non-semantic"-sources.

\subsection{All tree-buffer filters are now guarded against filtering out all tags.}
\label{sec-12.2}

   Each filter which a user can apply to the sources-, methods- or
   history-buffer is now guarded against filtering out all nodes by accident,
   ie. if a filter would empty the whole tree-buffer then this filter will be
   autom. unwinded.
   
\subsection{Much smarter mechanism to highlight the current tag in the methods-buffer.}
\label{sec-12.3}

   Previous versions of ECB have always fully expanded the whole tree in the
   Methods-buffer if the current tag in the source-buffer was not visible in
   the current tree - e.g. because the variables-bucket was collapsed or the
   containing type of a tag (e.g. the class of a method) was collapsed. So in
   most cases much more was expanded as needed to make the current tag
   visible.

   The mechanism of ECB 2.22 only expands the needed parts of the tree-buffer
   to make the related node visible: First we try to highlight the current tag
   with current expansion-state of the Methods-buffer. If the node is not
   visible so the tag can not be highlighted then we go upstairs the ladder of
   type-tags the current tag belongs to (e.g. we expand successive the nodes
   of the whole class-hierachy of the current method-tag until the related
   node becomes visible). If there is no containing type for the current tag
   then the node of the tag is probably contained in a toplevel-bucket which
   is currently collapsed; in this case we expand only this bucket-node and
   try highlighting again. Only if this has still no success then we expand
   the full tree-buffer and try to highlight the current tag.

   There is also a new option `ecb-auto-expand-tag-tree-collapse-other': If
   this option is set then auto. expanding the tag-tree collapses all not
   related nodes. This means that all nodes which have no relevance for the
   currently highlighted node will be collapsed, because they are not
   necessary to make the highlighted node visible. This feature is switched
   off by default because if always collapses the complete Methods-tree before
   the the following highlighting of the node for the current tag expands the
   needed parts of the tree-buffer.
   
\subsection{The popup-menus can now be nested into 4 levels of submenus. In general there}
\label{sec-12.4}

   could be an infinite depth of nesting but it makes no sense - if possible
   at all - to define infinite nested defcustom-types. So there is a limit of
   4 levels until a user asks for more - then this wish can be satisfied
   during 1 minute: Just increase the value of the `ecb-max-submenu-depth'
   before loading ECB. This enhancement affects the four options with name
   `ecb-*-menu-user-extension'.
   
\subsection{The indication for ECB in the minor-mode is now hidden when the ECB-windows}
\label{sec-12.5}

   are visible because in this case the activity of ECB is quite obvious. When
   the ECB-windows are hidden the the value of `ecb-minor-mode-text' is
   displayed in the modeline.

\subsection{Compatibility enhancements}
\label{sec-12.6}


\subsubsection{Cause of the fixed bugs the compatibility for `bs-show' has increased when}
\label{sec-12.6.1}

    the buffer-selection should be displayed in the permanent compile-window.

\subsection{Fixed Bugs}
\label{sec-12.7}


\subsubsection{Fixed a bug preventing the native Windows-port of XEmacs from working.}
\label{sec-12.7.1}


\subsubsection{Makes the behavior of adviced `display-buffer' fully compatible to the}
\label{sec-12.7.2}

    original version for all buffers which are neither ecb-tree-buffer nor
    compilation-buffers in the sense of `ecb-compilation-buffer-p'. These
    "source-buffers" are now also handled correctly in all situations.

\subsubsection{Fixed a small bug in `ecb-toggle-compile-window-height'. This bug has}
\label{sec-12.7.3}

    influenced the behavior of `display-buffer' when called from programs for
    "compilation-buffers".

\subsubsection{Fixed a bug related to the speedbar-integration.}
\label{sec-12.7.4}

    This bug complained "Can not switch buffer in dedicated window" when the
    intergrated speedbar-buffer was selected and the user tries to open a file
    or to jump to a tag.

\subsubsection{Fixed a bug for the special ECB-modeline-menu for the XEmacs-modelines.}
\label{sec-12.7.5}

    Now only in the special ECB-tree-buffer a special ECB-modeline-menu is
    bound to mouse-3, all other buffers display the standard modeline-menu of
    XEmacs.

\subsubsection{Fixed a bug which occurs with XEmacs in combination with func-menu.}
\label{sec-12.7.6}

    But there are still some incompatibilities between ECB and func-menu. So
    the recommendation is to disable func-menu-support when using ECB.
    Normally using func-menu makes no sense in combination with ECB because
    ECB provides the same and even more informations as func-menu - so
    func-menu is redundant ;-)
    


\section{Changes for ECB version 2.21}
\label{sec-13}


\subsection{Advice for `balance-windows' so only the edit-windows are balanced.}
\label{sec-13.1}

   Thanks to David Ishee <782zp6a02@sneakemail.com> for suggestion.

\subsection{New option `ecb-ignore-display-buffer-function'}
\label{sec-13.2}

   Per default the adviced version of `display-buffer' ignores the value of
   `display-buffer-function' when called for the ECB-frame. If this variable
   should not be ignored then the function of `display-buffer-function' is
   completely responsible which window is used for the buffer to display - no
   smart ECB-logic will help to deal best with the ECB-window-layout! You can
   define if and when `display-buffer-function' should be ignored: Always
   (default), when a compile-window is used or never.

\subsection{Compatibility enhancements}
\label{sec-13.3}

   In general the current layout-engine of ECB is so flexible that there
   should be no - or at least very few - conflicts between ECB and any other
   elisp-library - even when another lib is running during the ECB-windows are
   visible.

\subsubsection{Commands like `bs-show' of the library bs.el now work correctly with ECB}
\label{sec-13.3.1}


\subsubsection{Xrefactory works even when all ECB-windows are visible (see Fixed Bugs).}
\label{sec-13.3.2}


\subsubsection{Applications like Gnus or BBDB run withing the ECB-frame without conflicts -}
\label{sec-13.3.3}

    even when the ECB-windows are visible.

\subsubsection{Commands using `Electric-pop-up-window' now work correctly with ECB.}
\label{sec-13.3.4}

    This ensures that the electric-* commands (e.g. `electric-buffer-list' or
    `electric-command-history') work well with ECB. If the related
    "display-buffer" of such an electric command is a "compilation-buffer" in
    the sense of `ecb-compilation-buffer-p' then this buffer will be displayed
    in the compile-window of ECB - if there is any shown.

\subsection{For XEmacs the package fsf-compat is no longer required by ECB.}
\label{sec-13.4}

   But it is still necessary to check if fsf-compat is required by the
   packages semantic, eieio and speedbar which in turn are required by ECB.

\subsection{Fixed Bugs}
\label{sec-13.5}


\subsubsection{Fixed a fatal bug which prevents `other-window' from working with}
\label{sec-13.5.1}

    arguments < 0. This bug has also prevented Xrefactory from working correct
    when ECB is active - see compatibility enhancements above.

\subsubsection{If point stays in the current source-buffer on a function argument then still}
\label{sec-13.5.2}

    ECB highlights the function in the Methods-buffer.

\subsubsection{ECB now uses `compilation-window-height' correctly when set buffer-local}
\label{sec-13.5.3}

    as possible with latest CVS-version of GNU Emacs 21.

\subsubsection{Fixed a bug in `ecb-sources-filter' and `ecb-history-filter'.}
\label{sec-13.5.4}

    


\section{Changes for ECB version 2.20}
\label{sec-14}


\subsection{New customization group "ecb-most-important"}
\label{sec-14.1}

   This group centralizes the most important options of ECB in one group which
   is also offered by the "Preferences" submenu ob the ECB-menu. These are
   options you should at least know that they exist.

\subsection{The keybinding for the online-help has changed from [C-c . o] to [C-c . h]}
\label{sec-14.2}

   This has been done because the key [C-c . o] is needed for the new command
   `ecb-toggle-scroll-other-window-scrolls-compile'.

\subsection{The options `ecb-major-modes-activate' and `ecb-majors-mode-deactivate' have}
\label{sec-14.3}

   been replaced by one new option `ecb-major-modes-show-or-hide'. The purpose
   of the old option is now quite out-of-date because:
   - ECB supports window-managers like winring.el or escreen.el very well,
   - as of version 2.20 ECB has no restrictions about the number of
     edit-windows and
   - the new option simplifies things a lot.
      
\subsection{The restriction of only two edit-windows has been gone!}
\label{sec-14.4}


\subsubsection{Beginning with this version 2.20 there are no restrictions about the}
\label{sec-14.4.1}

    window-layout in the edit-area of ECB. This means you can split the
    edit-area of ECB in as many windows as you like. The split-state of the
    edit-area will be preserved when toggling the visibility of the
    ECB-windows and even between activation and deactivation of ECB. Deleting
    of certain edit-windows will never destroy the special ECB-windows or the
    compile-window. Just work with the edit-area of ECB as if it would be an
    extra frame!

\subsubsection{Option `ecb-split-edit-window' has been renamed in}
\label{sec-14.4.2}

    `ecb-split-edit-window-after-start' because this new name reflects much
    better the purpose of this option. In addition there is offered a new
    value 'before-deactivation which is also the new default value. This new
    value preserves the full state between activations of ECB, i.e. the
    visibility of the ECB-windows, the visibility of a compile-window and also
    the full split-state of the edit-area. ECB auto. upgrades your setting!

\subsubsection{Compile-window can now be displayed even when the ECB-windows are hidden.}
\label{sec-14.4.3}

    So now you can have the same compile-window functionality when the
    ECB-window are hidden as when the ECB-windows are visible. The state of
    the compile-window will be preserved when toggling the ecb-windows or when
    maximizing one ecb-windows! So you have the advantage of one special
    window for all help-, grep or compile-output also when the ecb-windows are
    hidden - a window which will not be deleted if you call
    `delete-other-windows' (bound to [C-x 1]) for one of the edit-windows.
    
\subsubsection{The option `ecb-primary-mouse-jump-destination' has been renamed to}
\label{sec-14.4.4}

    `ecb-mouse-click-destination' and has also changed its default value to
    'last-point. ECB autom. upgrades the old settings.

\subsubsection{New keybinding [C-c . g l] for selecting the last selected edit-window.}
\label{sec-14.4.5}

    Also available via the ECB-menu. The command is
    `ecb-goto-window-edit-last'.

\subsubsection{Option `ecb-other-window-jump-behavior' has been renamed to}
\label{sec-14.4.6}

    `ecb-other-window-behavior' and has also two new allowed values: 'smart
    (the new default) and an arbitrary function-symbol. With the latter one
    the user can define his own other-window-behavior and the former one
    chooses in a smart and intuitive way the "other window" for going to it
    via `other-window' or for scrolling it via one of the "scroll another
    window"-functions (e.g. `scroll-other-window'). Thanks to John S. Yates,
    Jr. <john@yates-sheets.org> for suggesting the new smart behavior. ECB
    autom. upgrades the old value of `ecb-other-window-jump-behavior' to the
    new option-name.
    
\subsubsection{New option `ecb-scroll-other-window-scrolls-compile-window' and new}
\label{sec-14.4.7}

    command `ecb-toggle-scroll-other-window-scrolls-compile' (bound to [C-c .
    o]). If this new option is nil then ECB chooses very smart and intuitive
    the window which will be scrolled by commands like `scroll-other-window'
    (see documentation of the the option `ecb-other-window-behavior'). But
    sometimes the user wants to scroll the compile-window from another window.
    With this new command the user can fest and easy toggle the behavior ECB
    chooses another window for scrolling.

\subsubsection{Higher compatibility of ECB with other packages}
\label{sec-14.4.8}

    Without the 2-edit-window-restriction ECB is now more compatible with
    other packages. For example the package calculator.el can now also being
    used without setting `calculator-electric-mode' to not nil - regardless in
    how many edit-windows the edit-area of ECB is splitted. Also the package
    bs.el (command `bs-show') benefits from the new layout-flexibility. As of
    ECB-version 2.20 there should be no conflicts between this package and
    ECB.

\subsection{Enhancements to the tree-buffers of ECB}
\label{sec-14.5}

    
\subsubsection{Enhancements to the popup-menus of the tree-buffers}
\label{sec-14.5.1}


\begin{itemize}

\item All popup-menus of the tree-buffers can be used with the tmm-library\\
\label{sec-14.5.1.1}

     The already existing command `tree-buffer-show-menu-keyboard' (bound to
     [M-m] in every tree-buffer of ECB) accepts now a prefix argument. If called
     with a prefix-argument (hit [C-u M-m]) then the popup-menu is displayed
     with the tmm-mechanismus (like the Emacs-[menu-bar] is displayed when
     `tmm-menubar' is called). Thanks for suggestion to Yvonne Thomson
     <yvonne@thewatch.net>.


\item Select an edit-window via popup where a source should be opened or a token\\
\label{sec-14.5.1.2}

     should be displayed.
    

\item The popup-menus can be dynamically extended.\\
\label{sec-14.5.1.3}

     See the new options `ecb-directories-menu-user-extension-function',
     `ecb-sources-menu-user-extension-function',
     `ecb-methods-menu-user-extension-function' and
     `ecb-history-menu-user-extension-function'.


\item All popup-menu-commands respect the setting of the option\\
\label{sec-14.5.1.4}

     `ecb-mouse-click-destination' (formerly known as
     `ecb-primary-mouse-jump-destination' - see above).

\end{itemize} % ends low level
\subsubsection{ECB supports the default modeline-mechanisms for deleting other windows.}
\label{sec-14.5.2}

    GNU Emacs binds [mouse-2] in its modeline to `delete-other-window'. ECB
    now supports this mechanism by binding a toggle-command to [mouse-2] in
    the modeline of each tree-buffer: If all ECB-windows are visible then this
    command maximizes the current tree-window and if current tree-window is
    maximized all ECB-windows are displayed again. XEmacs binds a popup-menu
    with some window commands to [button-3] in its modeline. ECB supports this
    mechanism by replacing this menu by a menu which offers exactly 2
    commands: Maximizing current tree-window and displaying all ECB-windows.

\subsubsection{Now all commands for selecting a tree-buffer work even when this buffer is}
\label{sec-14.5.3}

    not visible because another tree-buffer is maximized. If the tree-buffer
    is contained in the layout then the layout is redrawn with all its
    tree-buffers visible so that tree-buffer can be selected. Same for
    maximizing. An example for such a command is `ecb-goto-window-methods'.

\subsubsection{New option `ecb-maximize-ecb-window-after-selection'.}
\label{sec-14.5.4}

    Automatic maximizing of a tree-window after selecting it. Thanks for
    suggestion to John S. Yates, Jr. <john@yates-sheets.org>.
       
\subsubsection{Changes to the image-icons of the tree-buffers}
\label{sec-14.5.5}


\begin{itemize}

\item All image-files have been renamed from <IMAGENAME>.xpm to\\
\label{sec-14.5.5.1}

     ecb-<IMAGENAME>.xpm. This was necessary because the icons for "open" and
     "close" in Emacs toolbar got superseeded by ECB's icons for "open" and
     "close". The problem was, that ECB was using the same names for the icon
     files "open.xpm" and "close.xpm" as emacs toolbar does. This problem
     occurs if the image-directory of ECB is contained in the `load-path'
     which is when ECB is installed in the site-lisp directory of Emacs with
     the default subdirs.el of Emacs which adds automatically all
     subdirectories to the load-path (and therefore also subdirectories which
     does not contain elisp-files but only images - btw: not really smart;-).
     Thanks to Roland Schaeuble <roland.schaeuble@siemens.com> for pointing
     out that.


\item Added all image-icons in a 10-point height.\\
\label{sec-14.5.5.2}

     Thanks to Nick Cross <nick@goots.org> for sizing down the images.

\end{itemize} % ends low level
\subsection{New default-value for `ecb-post-process-semantic-taglist'.}
\label{sec-14.6}

   Now for buffers with major-mode `c-mode' all function-prototypes are
   filtered out. For example this is useful for editing C files which have the
   function prototypes defined at the top of the file and the implementations
   at the bottom. This means that everything appears twice in the methods
   buffer, but probably nobody wants to jump to the prototypes, they are
   only wasting space in the methods buffer. Of course in C-header-files
   function-prototypes will be displayed!

   Thanks for suggestion and a first implementation to Rob Walker
   <rob.lists@tenfoot.org.uk>.

\subsection{Changes for the compile-window}
\label{sec-14.7}

   
\subsubsection{The command `ecb-toggle-compile-window' now displays a compile-window even if}
\label{sec-14.7.1}

    the option `ecb-compile-window-height' is nil. This is a shortcut for
    customizing this option via the customize-feature if you just want
    displaying a compile-window for the current Emacs-session. In this
    situation ECB asks for the height of the compile-window, sets this height
    as new value of `ecb-compile-window-height' and displays the
    compile-window.

\subsubsection{New option `ecb-change-layout-preserves-compwin-state'.}
\label{sec-14.7.2}

    Changing the layout preserves now the state of the compile-window if this
    option is not nil. This is for example useful if the user toggles between
    several layouts and wants to preserve the hidden-state of the
    compile-window. Thanks to John S. Yates, Jr. <john@yates-sheets.org> for
    suggestion.

\subsubsection{Adviced `delete-window' and `delete-other-windows' handle compile-window.}
\label{sec-14.7.3}

    The former one hides the compile-window if called in or for that window
    whereas the latter one hides the compile-window if called in or for an
    unsplitted edit-window. Thanks to John S. Yates, Jr.
    <john@yates-sheets.org> for the suggestion.

\subsection{New option `ecb-activation-selects-ecb-frame-if-already-active'.}
\label{sec-14.8}

   See the docstring for more details. Thanks for suggestion to Jeff Jensen
   <jeffjensen@nospam.visi.com>

\subsection{The command `ecb-redraw-layout' can be called with prefix-argument(s).}
\label{sec-14.9}

   Called with ONE prefix-argument the command behaves as with ECB-versions <
   2.20, i.e. redrawing the full layout regardless of the current
   visibility-state of the ecb-windows or the compile-window. Called without a
   prefix-arg this command now preserves the state of the ecb-windows as well
   as the state of the compile-window. Called with TWO prefix-args means an
   emergency-redraw (see documentation of `ecb-redraw-layout').

\subsection{New command `ecb-display-news-for-upgrade'.}
\label{sec-14.10}

   ECB autom. displays after an upgrade a short summary of the most important
   NEWS. With the new command you can do this also on demand.
      
\subsection{Fixed Bugs}
\label{sec-14.11}


\subsubsection{Adviced `display-buffer' handles `same-window-buffer-names' and}
\label{sec-14.11.1}

    `same-window-regexps' correctly.

\subsubsection{Fixed a bug preventing tree-buffers with expand-symbol "before" to work}
\label{sec-14.11.2}


\subsubsection{Fixed a bug in the adviced version of `other-window'.}
\label{sec-14.11.3}

    Now it works correctly also when a durable compile-window is set but
    currently hidden.

\subsubsection{Fixed a bug when scrolling a window from within the minibuffer.}
\label{sec-14.11.4}

    For example this bug has prevented scrolling the *Completion* buffer from
    within the minibuffer. Now `minibuffer-scroll-window' is set correctly.

\subsubsection{Toggling the compile-window has not worked when `ecb-compile-window-height'}
\label{sec-14.11.5}

    has not been saved for future sessions. This is fixed now.

\subsubsection{`ecb-cycle-through-compilation-buffers' works with hidden compile-window.}
\label{sec-14.11.6}


\subsubsection{Fixed a bug when an ECB-window was maximized and the user has stored sizes}
\label{sec-14.11.7}

    for this layout.

\subsubsection{Fixed a bug in `ecb-major-modes-activate' and `ecb-major-modes-deactivate'}
\label{sec-14.11.8}


\subsubsection{Fixed a bug in the handling of the option `ecb-kill-buffer-clears-history'}
\label{sec-14.11.9}


\subsubsection{Fixed a bug which prevents some older version of XEmacs 21.4 working well.}
\label{sec-14.11.10}

    For XEmacs 21.4.6 a bug was reported which causes this XEmacs-version to
    fail when `compile' or `grep' is called. The reason for this was that
    these old versions of XEmacs 21.4 contain versions of `display-buffer'
    and/or `get-frame-for-buffer' which do not have the fourth argument
    SHRINK-TO-FIT. Now ECB has fixed this and handles correctly the option
    `temp-buffer-shrink-to-fit' and all shades of displaying and compiling
    buffers even for these versions of XEmacs.

    

\section{Changes for ECB version 2.11}
\label{sec-15}


\subsection{Using semanticdb to jump to type-tags defined in other files.}
\label{sec-15.1}

   In OO-languages like CLOS, eieio and C++ there can be type-tags in the
   method-buffer which are somehow virtual because there is no definition in
   the current source-file. But such a virtual type collects all its outside
   defined members like methods in C++ which are defined in the *.cc file
   whereas the class-definition is defined in the associated header-file.
   ECB uses semanticb to open the definition-file of such a tag and to jump to
   the definition of this tag. Same for parent-tags in the methods-buffer.
   This feature can only work correctly if semanticdb is well configured!

\subsection{New option `ecb-ignore-special-display'}
\label{sec-15.2}

   The options `special-display-buffer-names', `special-display-regexps' and
   `special-display-function' work as expected. Per default this in only true,
   when no durable compile-window is used (see option
   `ecb-compile-window-height'), i.e. with a durable compile window the
   special-display-options are ignored per default. But this behavior can be
   changed with the new option `ecb-ignore-special-display'.

   Thanks to Rob Walker <rob.lists@tenfoot.org.uk> for a first suggestion.

\subsection{Better reporting if there are errors during ECB-startup}
\label{sec-15.3}


\subsection{Fixed bugs}
\label{sec-15.4}


\subsubsection{Clicking onto nodes with positionless semantic-tags as data doesn't fail}
\label{sec-15.4.1}


\subsubsection{Non-semantic sources (e.g. Perl, TeX) work when cedet 1.0 is loaded}
\label{sec-15.4.2}


\subsubsection{Fixed the versioning of ECB so the autom. upgrade works}
\label{sec-15.4.3}




\section{Changes for ECB version 2.01}
\label{sec-16}


\subsection{Prepared for the new semantic 2.0 beta (contained in the cedet 1 beta)}
\label{sec-16.1}


\subsubsection{New naming convention}
\label{sec-16.1.1}


    Forthcoming semantic 2.0 (part of the new cedet-library) introduces a new
    naming convention. Here is the relevant part of the semantic 2.0
    NEWS-file:
    
    token     - Refers to a lexical analyzer production.
    stream    - A list of tokens
    tag       - Refers to a datastructure created by a grammar to represent
                something in a language file
    table     - A hierarchical list of tags.
    tag-class - A tag may represent a function, data type, or variable.
    parse     - Run a source file through a parser.
    
    Normally there is no need for ECB users to bother with all details of
    semantic naming conventions but for the sake of consisteny between
    semantic and ECB the ECB-package has renamed all its variables and
    functions from "token" to "tag". Therefore also all options which contain
    "token" in their name have been renamed by replacing "token" with "tag".
    See RELEASE_NOTES for a list of this options.
 
    ECB autom. upgrades all old-values of these options to the new options!
    There is nothing to do for you.

\subsubsection{ECB works with new semantic 2.0 which is shipped within cedet 1.0}
\label{sec-16.1.2}


\subsection{Overhaul of the display-layer and -handling of the tree-buffers}
\label{sec-16.2}


\subsubsection{Now there are 3 different styles available}
\label{sec-16.2.1}

    * Image: Looks very nice and modern - just give it a try or see the
      screenshots-section at [[http://ecb.sourceforge.net][http://ecb.sourceforge.net]]
    * Ascii with guide-lines: Drawing the trees with ascii-symbols
    * Ascii without guide-lines: This is the style of ECB <= 1.96

    For details and a visualization of these different styles see the new
    option `ecb-tree-buffer-style' (replaces the old option
    `ecb-tree-use-image-icons').

    Especially the new image-style was inspired by the library tree-widget.el
    written by David Ponce <david@dponce.com>. Also some images shipped with
    ECB are "stolen" from this library.

\subsubsection{No-leaf-tree-nodes with currently no subnodes are displayed with [x]}
\label{sec-16.2.2}

    An example are directories in the Directories tree-buffer which have
    currently no subdirectories. Leafs like source-files or methods are of
    course handled as leafs and not as currently empty nodes.

\subsubsection{Fixed some inconsistencies in expanding, collapsing and selecting nodes}
\label{sec-16.2.3}


\subsubsection{Moving all options related to the tree-buffer-style or -handling to the new}
\label{sec-16.2.4}

    customize-group "ecb-tree-buffer".

\subsection{Completely rewritten popup-menu mechanism.}
\label{sec-16.3}


\subsubsection{Now sub-menus are allowed for all popups}
\label{sec-16.3.1}

    To give a better lucidity of the popup-menus of the tree-buffers these
    menus can now being arranged in sub-menus. The default values now already
    use sub-menus. So if you have added menu-entries to one of the options
    `ecb-directories-menu-user-extension', `ecb-sources-menu-user-extension',
    `ecb-methods-menu-user-extension' or `ecb-history-menu-user-extension' ECB
    resets these options to the new defaults of ECB and adds at the top your
    "old" personal entries. This comes because the type of these options has
    changed. Thanks for general suggestion to Ole Arndt <ole@sugarshark.com>.

\subsubsection{Added new defaults to `ecb-directories-menu-user-extension'}
\label{sec-16.3.2}

    Three new entries "CVS status", "CVS examine" and "CVS update" in a new
    sub-menu "Version control" for running CVS-commands against the directory.
    These menu-entries are added as new default values to
    `ecb-directories-menu-user-extension' so a user can delete or change them
    if he does not use CVS but another revision-tool (e.g. Clearcase). Thanks
    for suggestion and first implementation to Ole Arndt <ole@sugarshark.com>.

\subsubsection{Added new defaults to `ecb-sources-menu-user-extension'}
\label{sec-16.3.3}

    One new sub-menu "Version control" with some senseful commands. See also
    new default for the option `ecb-directories-menu-user-extension'. Same is
    added to the `ecb-history-menu-user-extension'. Thanks for suggestion and
    first implementation to Ole Arndt <ole@sugarshark.com>.

\subsection{Possibility to define which types should be expanded at file-open-time}
\label{sec-16.4}

   Semantic groups types into different type-specifiers. Current available
   type-specifiers are for example "class", "interface", "struct", "typedef",
   "union" and "enum". With the new option `ecb-type-tag-expansion' you can
   specify on a major-mode-basis which type-specifiers should be expanded at
   file-open-time and which not. So for example in C++ it could be senseful
   not to expand the types with a type-specifiers "typedef", "struct" or
   "enum" (see the default-value of this option).
    
\subsection{Adding a filter-feature to the Sources- and the History-buffer}
\label{sec-16.5}

   Now a filter can be applied to the Sources- and/or History-buffer which
   desides which entries are displayed in these buffers. See the new commands
   `ecb-sources-filter' (bound to [C-c . fs]) and `ecb-history-filter' (bound
   to [C-c . fh]) and the new entries for the popup-menus for these buffers.
   In the sources-buffer each directory has its own filter. The currently
   applied filter is displayed in the modeline of that tree-buffer with the
   new face `ecb-mode-line-prefix-face' (s.b.)

\subsection{Changes related to the modelines of the tree-buffers}
\label{sec-16.6}

   
\subsubsection{Added faces to the mode-lines of the ECB-tree-buffers.}
\label{sec-16.6.1}

    See the three new options `ecb-mode-line-win-nr-face',
    `ecb-mode-line-prefix-face' and `ecb-mode-line-data-face' and also they
    related (and equally named) new faces.

\subsubsection{Moved all options related to the modelines in the new customize-group}
\label{sec-16.6.2}

    "ecb-mode-line".
       
\subsection{No window-restrictions if the ECB-windows are hidden}
\label{sec-16.7}

   If the special ECB-windows are hidden (e.g. by `ecb-toggle-ecb-windows')
   then there are no restrictions about the window-layout of the ecb-frame:
   The frame can be splitted in any arbitrary windows. All adviced functions
   behave as their originals. So the frame can be used as if ECB would not be
   active but ECB IS still active in the "background" and all ECB-commands and
   all ECB-keybindings can be used. Of course some of them doesn't make much
   sense but nevertheless they can be called.

   Therefore it should be enough to hide the ECB-windows to run other
   Emacs-applications which have their own window-layout-managing. There
   should be no conflicts. But nevertheless the most recommended method for
   running ECB and other applications (e.g. xrefactory, Gnus etc.) in the same
   frame is to use a window-manager like winring.el or escreen.el!

\subsection{Ediff runs per default in the ECB-frame.}
\label{sec-16.8}

   Now ediff can run per default in the ECB-frame because ECB ensures that all
   special ECB-windows are hidden before ediff sets up its own window-layout.
   ECB also restores exactly the "before-ediff" window-layout of the ecb-frame
   See new option `ecb-run-ediff-in-ecb-frame'.

\subsection{More flexible directory-caching with new `ecb-cache-directory-contents-not'}
\label{sec-16.9}

      
\subsection{New option `ecb-advice-window-functions-signal-error'}
\label{sec-16.10}

   Now the adviced window functions of `ecb-advice-window-functions' do not
   signal per default an error if called in situations which are not allowed -
   they simple do nothing. An example is calling `delete-window' in a special
   ecb-window or in the compile-window; if this new option is nil then nothing
   is done otherwise an error is signaled. If you want the old behavior of
   signaling an error just set this new option to not nil.

\subsection{Fixed Bugs}
\label{sec-16.11}


\subsubsection{When the special ECB-windows were hidden and a durable compile-window was set}
\label{sec-16.11.1}

    then ECB shows the special ECB-windows when a temp-buffer (e.g.
    *Help*-buffers) has been displayed. This annoying behavior is fixed.

\subsubsection{XEmacs has not added the ECB-menu to the menubar of that buffers which are}
\label{sec-16.11.2}

    already alive while activating ECB. On the other side XEmacs has not
    removed the ECB-menu from the menubar for all living buffers after
    deactivating ECB. Both of these two bugs are fixed now.

\subsubsection{JDEE-dialogs now work correctly.}
\label{sec-16.11.3}

    JDEE offers sometimes "dialogs" where the user can choose among several
    options. These "dialogs" now work correctly regardless if a durable
    compile-window is used or not.

\subsubsection{A bug in displaying the tree-buffers in a different font or font-height}
\label{sec-16.11.4}




\section{Changes for ECB version 1.96}
\label{sec-17}


\subsection{ECB can work together with the window-managers escreen and winring}
\label{sec-17.1}

   This allows to run applications like Gnus, VM or BBDB in the same frame as
   ECB! See the new lib ecb-winman-support.el and read the section
   "Window-managers and ECB" in the chapter "Tips and tricks" in the
   info-manual of ECB.
   
   Thanks to Johann "Myrkraverk" Oskarsson <myrkraverk@users.sourceforge.net>
   who has reported some bugs related to escreen and has therefore motivated a
   lot to add support for the well known window-managers escreen and winring.

\subsection{ECB can display the window-number in the modeline of the special windows.}
\label{sec-17.2}

   The left-top-most window in a frame has the window-number 0 and all other
   windows are numbered with increasing numbers in the sequence, functions
   like `other-window' or `next-window' would walk through the frame. This
   allows to jump to windows by name as described at the Emacs-Wiki
   [[http://www.emacswiki.org/cgi-bin/wiki.pl?SwitchingWindows][http://www.emacswiki.org/cgi-bin/wiki.pl?SwitchingWindows]].

   See the new option `ecb-mode-line-display-window-number'. Currently this
   feature is only available for GNU Emacs 21.X, because neither GNU Emacs <
   21 nor XEmacs can dynamically evaluate forms in the mode-line.

   Thanks to Johann "Myrkraverk" Oskarsson <myrkraverk@users.sourceforge.net>
   for suggesting this.

   In addition the already existing options `ecb-mode-line-data' and
   `ecb-mode-line-prefixes' are now more flexible and are now able to define
   special modelines not only for the 4 builtin ECB-tree-buffers but for every
   special ECB-buffer. ECB will autom. upgrade the old values of these options
   as best as possible to their new types!
               
\subsection{Much better support of the ECB-compile-window}
\label{sec-17.3}


\subsubsection{General overhaul for displaying buffers in the ECB-compile-window.}
\label{sec-17.3.1}

    The whole mechanism has been rewritten so now using a durable
    compile-window is much safer and works as well as without one. Now all
    buffers displayed with `display-buffer' are checked if they are
    "compilation-buffers" in the sense of the function
    `ecb-compilation-buffer-p' and then displayed in the compile-window of
    ECB. Especially temp-buffers displayed with `with-output-to-temp-buffer'
    (e.g. *Help*-buffers) are now much better supported.

    Simply said: All buffers for which `ecb-compilation-buffer-p' says t are
    now *always* displayed in the compile-window (if there is any) with
    complete consideration of Emacs-options like `compilation-window-height',
    `temp-buffer-max-height', `temp-buffer-resize-mode' (GNU Emacs),
    `temp-buffer-shrink-to-fit' (XEmacs) etc. All other buffer not.

\subsubsection{New layout-option `ecb-compile-window-width'.}
\label{sec-17.3.2}

    Thanks for suggestion to John S. Yates, Jr. <jyates@netezza.com>

\subsubsection{New layout-option `ecb-compile-window-prevent-shrink-below-height'}
\label{sec-17.3.3}


\subsubsection{Better and more reliable eshell-integration. Now the compile-window fits}
\label{sec-17.3.4}

    always autom. to the output of the last eshell-command.

\subsubsection{All options related to the ECB-compile-window have been moved to the new}
\label{sec-17.3.5}

    customization-group "ecb-compilation" which is a subgroup of "ecb-layout".
    
\subsection{Safer handling of all adviced functions of ECB}
\label{sec-17.4}

   All ECB-advices are now active if and only if ECB is running. Just loading
   the ECB-library doesn't activate any ECB-advice! Also if there is an error
   during the activation-process of ECB then all advices are always disabled
   automatically.

\subsection{ECB can visit tokens in a smarter way}
\label{sec-17.5}


\subsubsection{Narrowing via popup-menu of the Methods-buffer includes now the documentation}
\label{sec-17.5.1}

    of the token if located outside of the token (e.g. Javadoc for Java).

\subsubsection{More flexible definition what to do after visiting a token via the}
\label{sec-17.5.2}

    Methods-buffer. See the new option `ecb-token-visit-post-actions' which now
    replaces the options ecb-highlight-token-header-after-jump',
    `ecb-scroll-window-after-jump' and `ecb-token-jump-narrow'.

\subsection{ECB selects autom. the current default-directory after activation even if}
\label{sec-17.6}

   no file-buffer is displayed in the edit-window. This is useful if ECB is
   autom. activated after startup of Emacs and Emacs is started without a
   file-argument. So the directory from which the startup has performed is
   auto. selected in the ECB-directories buffer and the ECB-sources buffer
   displays the contents of this directory. See new option
   `ecb-display-default-dir-after-start'. 

\subsection{ECB preserves the split-state of the frame after activation rsp. deactivation}
\label{sec-17.7}

   This is done if the already existing option `ecb-split-edit-window' has the
   new value t.
      
\subsection{New command `ecb-toggle-window-sync' for fast toggling between autom.}
\label{sec-17.8}

   synchronization of the ECB-buffers is switched on and off.

\subsection{The popup-menu of the Directories-buffer allows opening a dir with Dired.}
\label{sec-17.9}

   The Sources-Buffer offers the same for the current selected directory. In
   the History-buffer the directory of the selected file-buffer can be
   "dired".

\subsection{Now manually resizing of the ECB-windows is possible even if}
\label{sec-17.10}

   `ecb-fix-window-size' is not nil for current layout. Thanks for suggestion
   Massimiliano Mirra <mmirra@libero.it>. Only relevant for GNU Emacs 21.X.

\subsection{ECB does not call any functions of the cl-library at runtime.}
\label{sec-17.11}


\subsection{Fixed Bugs}
\label{sec-17.12}


\subsubsection{Fixed all the file- and directory-commands in the popup-menus.}
\label{sec-17.12.1}

    Now the Directories- and Sources-buffers are autom. updated after creating
    rsp. deleting source-files or directories.

\subsubsection{Jumping to some tokens for Perl sources failed. This is fixed now.}
\label{sec-17.12.2}


\subsubsection{Now the buffer-names displayed in the History-window are always uptodate,}
\label{sec-17.12.3}

    even if file-buffers are killed/opened during hidden ECB-windows.

\subsubsection{Fixed a conflict between ECB and `tmm-menubar' (tmm.el)}
\label{sec-17.12.4}

    
\subsubsection{Fixed a lot of misspellings in the ECB-online-help and also in all comments}
\label{sec-17.12.5}

    and strings of the sources.

\subsubsection{Fixed a bug in `ecb-toggle-enlarged-compilation-window'.}
\label{sec-17.12.6}

    This function has been also renamed to `ecb-toggle-compile-window-height'.

\subsubsection{Fixed a bug in the layout-engine related to dedicated windows.}
\label{sec-17.12.7}

    Thanks to Johann "Myrkraverk" Oskarsson <myrkraverk@users.sourceforge.net>
    who helped to find this bug.

\subsubsection{Fixed some references in the info-manual of ECB.}
\label{sec-17.12.8}




\section{Changes for ECB version 1.95.1}
\label{sec-18}


\subsection{Now the ecb-windows can be "maximized", means all other ecb-windows are}
\label{sec-18.1}

   deleted so only the edit-window(s), and this maximized ecb-window are
   visible (and maybe a compile-window if active). This can be done either via
   the popup-menus of the ecb-windows or via calling `delete-other-windows'
   (bound to [C-x 1]) in one of the ecb-windows or via the new commands
   `ecb-maximize-window-*' where * stands for {directories, sources, methods,
   history, speedbar}. There is also a command
   `ecb-cycle-maximized-ecb-buffers'.

   *Note*: If there are byte-compiled(!) user-defined layouts - either created
   interactively by `ecb-create-new-layout' or programmed with the macro
   `ecb-layout-define' - then the file where these user-defined layouts are
   saved (see option `ecb-create-layout-file') must be re-byte-compiled with
   latest ECB version >= 1.95.1! If the user-defined layouts are not
   byte-compiled then there is nothing to do.

\subsection{Some default key-bindings have changed; see RELEASE_NOTES for details.}
\label{sec-18.2}


\subsection{All popup-menus of the ECB-tree-buffers can be opened via [Meta-m]}
\label{sec-18.3}


\subsection{Fixed bugs}
\label{sec-18.4}


\subsubsection{Clicking onto the image-icons of an integrated speedbar has failed for}
\label{sec-18.4.1}

    XEmacs. This comes from a bug in speedbar which is now fixed locally in
    ECB by advicing `dframe-mouse-set-point' if speedbar is integrated in the
    ECB-frame.



\section{Changes for ECB version 1.95}
\label{sec-19}


\subsection{ECB now displays the expand- and collapse symbols in the tree-buffers with}
\label{sec-19.1}

   image-icons - the same icons as speedbar uses. See the new option
   `ecb-tree-use-image-icons'.

\subsection{Adding hideshow to the popup-menu of the Methods-buffer.}
\label{sec-19.2}

   This popup-menu now offers two entries for hiding rsp. showing the block
   of that token in the Methods-buffer for which the popup-menu was opened.
   This is done with the hideshow.el library.
   Thanks to Christoff Pale <christoff_pale@yahoo.com> for suggestion.

\subsection{Horizontal scrolling of the tree-buffers by clicking the edges of the}
\label{sec-19.3}

   modeline with mouse-1 or mouse-2. I.e. if you click with mouse-1 onto the
   left (rsp right) edge of the modeline you will scroll left (rsp. right)
   with the scroll-step defined in `ecb-tree-easy-hor-scroll'. This is only
   for GNU Emacs because XEmacs has hor. scrollbars.
   
\subsection{Changed default key-bindings:}
\label{sec-19.4}

   - C-c . r:  `ecb-rebuild-methods-buffer'
   - C-c . lc: `ecb-change-layout'
   - C-c . lr: `ecb-redraw-layout'
   - C-c . lt: `ecb-toggle-layout'
   - C-c . lw: `ecb-toggle-ecb-windows'

\subsection{Starting ECB is now possible via the "Tools"-menu}
\label{sec-19.5}


\subsection{ECB now requires speedbar version 0.14beta1 or higher. An automatic}
\label{sec-19.6}

   requirements-check is done by ECB. This is because now ECB has three
   dependencies to speedbar-code: Integrating whole speedbar in the ECB-frame,
   using speedbar-logic to parse files with imenu or etags and using the icons
   of speedbar for the ECB-tree-buffers.

\subsection{Fixed bugs}
\label{sec-19.7}


\subsubsection{Now the "goto-window-..." menu-entries in the "ECB"-menu are working correct}
\label{sec-19.7.1}

    if a speedbar is integrated into the layout.

\subsubsection{Preventing etags-supported sources from being parsed (and saved) too often.}
\label{sec-19.7.2}




\section{Changes for ECB version 1.94}
\label{sec-20}


\subsection{Supporting of non-semantic-sources.}
\label{sec-20.1}


\subsubsection{Native parsing and displaying source-contents of imenu supported files}
\label{sec-20.1.1}


\subsubsection{Native parsing and displaying source-contents of etags supported files}
\label{sec-20.1.2}


\subsubsection{There are some new options like `ecb-process-non-semantic-files',}
\label{sec-20.1.3}

    `ecb-non-semantic-parsing-function' and `ecb-method-non-semantic-face' and
    one new face `ecb-method-non-semantic-face'. See new customize group
    "ecb-non-semantic"!
    
\subsection{Better speedbar integration into the ECB-frame}
\label{sec-20.2}


\subsubsection{Now speedbar can be used not only as replacement for the}
\label{sec-20.2.1}

    ECB-directory-browser in the directory-window but also instead of the
    sources- or the methods-buffer/window. See new option
    `ecb-use-speedbar-instead-native-tree-buffer' which replaces the old
    option `ecb-use-speedbar-for-directories'.

\subsubsection{Now speedbar can be integrated into an arbitrary layout in the same way as}
\label{sec-20.2.2}

    the other four special buffers (directories, sources, methods and
    history). See new example layout with name "left-dir-plus-speedbar".

\subsubsection{New option `ecb-directories-update-speedbar' which helps you}
\label{sec-20.2.3}

    to update a speedbar-window to the current selected directory of the
    ECB-directories-window.

\subsection{New option `ecb-compilation-predicates' for better specifying which buffers}
\label{sec-20.3}

   should be treated as compilation-buffers and therefore be displayed in the
   compile-window of ECB - if there is any.
   
\subsection{Better customizing of the mode-line of an ECB-tree-buffer.}
\label{sec-20.4}

   See the new option `ecb-mode-line-data'.

\subsection{Fixed bugs:}
\label{sec-20.5}


\subsubsection{Using "Add source path" from the popup-menu now doesn't open the Windows}
\label{sec-20.5.1}

    file-dialog-box onto Windows-systems because this would prevent this
    command from working.

\subsubsection{Smart arrowkey-navigation in the tree-buffers is now even smarter.}
\label{sec-20.5.2}

    Thanks to John Russel <jorussel@cisco.com> for suggestion.

\subsubsection{If ECB was activated with a layout which does not contain a}
\label{sec-20.5.3}

    directory-buffer then you got always an empty directory-buffer even after
    switching to a layout with a directory-buffer. This is fixed now.

    

\section{Changes for ECB version 1.93}
\label{sec-21}


\subsection{Fixed bugs:}
\label{sec-21.1}


\subsubsection{The adviced version of `other-window' now works correct too if a}
\label{sec-21.1.1}

    minibuffer-window is active.

\subsubsection{The commands of `winner-mode' are now only forbidden in the ECB-frame}
\label{sec-21.1.2}

    but work in any other frame.

\subsubsection{`scroll-all-mode' now works correct in other frames than the ECB-frame.}
\label{sec-21.1.3}


\subsubsection{Removed the annoying message "Partial reparsing..."}
\label{sec-21.1.4}


\subsection{Enhancements to the layout:}
\label{sec-21.2}


\subsubsection{Slightly better redrawing current layout especially if the layout contains}
\label{sec-21.2.1}

    a compile-window.

\subsubsection{ECB autom. displays hidden compile-window before starting a compilation.}
\label{sec-21.2.2}

    This is done if `ecb-compile-window-height' is not nil but the
    compile-window is currently hidden by `ecb-toggle-compile-window'. Of
    course this works also for grep and other compile-modes. Same for
    `switch-to-buffer' and `switch-to-buffer-other-window' if called for a
    compilation-buffer in the sense of `ecb-compilation-buffer-p'.

\subsubsection{`switch-to-buffer-other-window' now works more naturally within the}
\label{sec-21.2.3}

    ECB-frame, especially if a compile-window is used.

\subsubsection{`ecb-toggle-compile-window' now has a default key-binding: [C-c . \]}
\label{sec-21.2.4}


\subsubsection{Now also layouts with user-defined special ecb-windows can be created}
\label{sec-21.2.5}

    interactively. See the online-manual for further details.

\subsection{Enhancements to the tree-buffers:}
\label{sec-21.3}

   
\subsubsection{Now user-extensions can be added to the popup-menus of the tree-buffers.}
\label{sec-21.3.1}

    See new options `ecb-directories-menu-user-extension',
    `ecb-sources-menu-user-extension', `ecb-methods-menu-user-extension',
    `ecb-history-menu-user-extension'.

\subsubsection{The methods-buffer now has a popup-menu with senseful actions}
\label{sec-21.3.2}

    
\subsection{New option `ecb-sources-exclude-cvsignore' which allows to exclude files from}
\label{sec-21.4}

   being displayed in the sources-buffer if they are contained in a .cvsignore
   file.

\subsection{ECB now also works with buffers "online" extracted from archives.}
\label{sec-21.5}

   Buffers extracted from an archive in `tar-mode' or `archive-mode' are now
   correct handled as if they were normal file-buffers. This feature doesn't
   work with XEmacs cause of its tar-mode and arc-mode implementation which
   does not handle extracted files as normal files.

\subsection{ECB now checks at load-time if the required packages semantic and eieio are}
\label{sec-21.6}

   at least available - regardless of the needed version. If at least one of
   these packages is missing then ECB offers to download and install it. More
   details are available in the installation instructions.



\section{Changes for ECB version 1.92.1}
\label{sec-22}


\subsection{Fixed bugs:}
\label{sec-22.1}


\subsubsection{Fixing eating up a lot of CPU-power just for updating the menu-}
\label{sec-22.1.1}

    entries of current compilation-buffers in the ECB-menu. Now the mechanism
    is based on a idle-timer and a cache.
    Fixed with great testing-help by Dan Debertin <airboss@nodewarrior.org>

\subsubsection{Fixed a bug which always truncates lines in the ECB-windows regardless}
\label{sec-22.1.2}

    of the value of `ecb-truncate-lines'

\subsubsection{The upgrade library of ECB now explicitly requires and loads the}
\label{sec-22.1.3}

    library executable.el because Emacs 20.X does not autoload the
    function `executable-find'.

\subsubsection{`ecb-toggle-compile-window' now works also correct if called with}
\label{sec-22.1.4}

    a prefix arg.



\section{Changes for ECB version 1.92}
\label{sec-23}


\subsection{Fixed bugs:}
\label{sec-23.1}


\subsubsection{Fixed small bugs in the popup-menu of the history-buffer.}
\label{sec-23.1.1}


\subsubsection{Fixed some not working links in the FAQ.}
\label{sec-23.1.2}


\subsection{ECB now includes a file "ecb-autoloads.el" which contains all available}
\label{sec-23.2}

   autoloads of ECB. Therefore loading this file is enough.

\subsection{Enhancements to the layout}
\label{sec-23.3}


\subsubsection{The sizes of the ecb-windows can be fixed even during frame-resizing}
\label{sec-23.3.1}

    This new feature is only available with GNU Emacs 21.X. See new option
    `ecb-fix-window-size'.
    Suggested by Gunther Ohrner <G.Ohrner@post.rwth-aachen.de>
     
\subsubsection{The command `ecb-store-window-sizes' can now also store fixed sizes.}
\label{sec-23.3.2}

    If called with a prefix arg then fixed sizes are stored instead of
    fractions of frame-width rsp. -height.

\subsubsection{The command `split-window' is now also adviced to work properly.}
\label{sec-23.3.3}


\subsection{Enhancements for the tree-buffers}
\label{sec-23.4}


\subsubsection{"Create Source" in the popup-menus now work also for non-java-files.}
\label{sec-23.4.1}


\subsubsection{ecb-truncate-lines can now be set different for each ECB-tree-buffer.}
\label{sec-23.4.2}


\subsubsection{Easy horizontal mouse-scrolling of the ECB tree-buffers. See new option}
\label{sec-23.4.3}

    `ecb-tree-easy-hor-scroll'.

\subsection{Added a command `ecb-jde-display-class-at-point' which displays the contents}
\label{sec-23.5}

   of the java-class under point in the methods-buffer of ECB.

\subsection{Renamed previous HISTORY file to NEWS and reformated it for outline-mode.}
\label{sec-23.6}




\section{Changes for ECB version 1.91.1}
\label{sec-24}


\subsection{Fixed bugs:}
\label{sec-24.1}


\subsubsection{Fixes a bug in the speedbar integration which prevented clicking onto a}
\label{sec-24.1.1}

    directory in the speedbar: Always after clicking onto a directory speedbar
    has re-synced itself to the directory of the current source-buffer in the
    edit-window. This bug is fixed, now directory navigation in the speedbar
    works. (Klaus)

\subsubsection{Fixes a bug in upgrading stored window-sizes from ECB 1.80 to ECB >= 1.90.}
\label{sec-24.1.2}

    Fixes also two bugs in `ecb-store-window-sizes' if default-values are used in
    `ecb-layout-window-sizes' or if ecb-windows have frame-height as height
    (rsp. frame-width as width) . (Klaus)

\subsubsection{Fixed a bug in the navigation-history of ECB which sometimes has prevented}
\label{sec-24.1.3}

    that a user can open files from the sources/history buffer or clicking onto
    tokens in the method-buffer. Now the back-forward-navigation works stable.
    (Klaus)
    
\subsection{Enhanced the downloading feature of ECB}
\label{sec-24.2}


\subsubsection{Fixed a bug in the downloading feature of ECB. (Klaus)}
\label{sec-24.2.1}


\subsubsection{Now ECB autom. checks at the download sites which versions are available.}
\label{sec-24.2.2}

    With the new option `ecb-download-package-version-type' you can define
    which types of version you allow to download (only stable, stable and betas
    or stable, betas and alphas); see related section in the online-manual!
    (Klaus)

\subsubsection{New command `ecb-download-semantic' for easy getting latest semantic}
\label{sec-24.2.3}

    versions. (Klaus)



\section{Changes for ECB version 1.90}
\label{sec-25}


\subsection{The website of ECB has moved from \href{http://home.swipnet.se/mayhem/ecb.html}{http://home.swipnet.se/mayhem/ecb.html} to}
\label{sec-25.1}

   \href{http://ecb.sourceforge.net}{http://ecb.sourceforge.net}. Maintainance has switched from Jesper Nordenberg
   to Klaus Berndl.

\subsection{Fixed bugs:}
\label{sec-25.2}

   
\subsubsection{Fixed an annoying bug which results in an error "Wrong type argument:}
\label{sec-25.2.1}

    integer-or-marker-p nil)'' after a full buffer reparse. ECB 1.80 has repaired
    its internal state but nevertheless the user had to re-click on the same
    token after this error to really jump to a token. With ECB 1.90 this bug has
    been gone (Klaus).

\subsubsection{Fixed a small bug in `ecb-add-all-buffers-to-history': If}
\label{sec-25.2.2}

    `ecb-sort-history-items' is nil then now the most recently used buffers are
    added to the top of the history, the seldom used buffers at the bottom.
    Thanks to Stefan Reich��r <xsteve@riic.at>! (Klaus).

\subsubsection{Fixed some bugs concerning the eshell-integration (Klaus):}
\label{sec-25.2.3}


\begin{itemize}

\item Synchronizing does not set the mark anymore\\
\label{sec-25.2.3.1}



\item No longer synchronizing after every command if current source resides in\\
\label{sec-25.2.3.2}

     the HOME directory.


\item Redrawing the layout preserves the eshell-buffer in the compile-window.\\
\label{sec-25.2.3.3}



\item Standard emacs-hooks are not clobbered with ecb-eshell-functions if ECB is\\
\label{sec-25.2.3.4}

     not active.

\end{itemize} % ends low level
\subsection{General enhancements:}
\label{sec-25.3}

     
\subsubsection{Now ECB displays at every start a "Tip of the day" in a dialog-box. This can}
\label{sec-25.3.1}

    be switched off with option `ecb-tip-of-the-day'. There is also a new
    command `ecb-show-tip-of-the-day' (Klaus).

\subsubsection{Advicing `scroll-all-mode' so it works correct with ECB, means}
\label{sec-25.3.2}

    `scroll-all-mode' scrolls only the windows of the edit-area, i.e. max. the
    two edit-windows (Klaus).

\subsubsection{Autom. preventing `winner-mode' from being activated as long as ECB is}
\label{sec-25.3.3}

    running (Klaus).

\subsubsection{Now autom. (de)activating ECB via the options ecb-major-modes-activate and}
\label{sec-25.3.4}

    ecb-major-modes-deactivate also works when opening files via dired (Klaus).

\subsubsection{Now ECB can be "autoloaded", i.e.}
\label{sec-25.3.5}

\begin{itemize}
\item ecb-activate,
\item ecb-minor-mode,
\item ecb-byte-compile and
\item ecb-show-help
\end{itemize}
    are autoload-able via `(autoload \ldots{})' (Klaus)

\subsubsection{Silencing the byte-compiler, i.e. byte-compilation of ECB should now be}
\label{sec-25.3.6}

    warning free. If byte-compilation still throws warnings during
    byte-compilation with `ecb-byte-compile' (rsp. using the Makefile or
    make.bat) please send this to our mailing list. (Klaus)

\subsection{Enhancements to the methods/variable/token tree-buffer:}
\label{sec-25.4}

   
\subsubsection{New feature: Now the methods buffer is auto. expanded if the node related to}
\label{sec-25.4.1}

    the current token in the edit-window is not visible (probably because its
    parent is collapsed). See new options `ecb-auto-expand-token-tree' and
    `ecb-expand-methods-switch-off-auto-expand' and the new command
    `ecb-toggle-auto-expand-token-tree' (Klaus).
  
\subsubsection{New command `ecb-expand-methods-nodes' which allows precisely expanding}
\label{sec-25.4.2}

    tokens with a certain indentation-level. There are also two new related
    options `ecb-methods-nodes-expand-spec' and
    `ecb-methods-nodes-collapse-spec' (Klaus).

\subsection{Enhancements to the directories tree-buffer:}
\label{sec-25.5}

    
\subsubsection{New command `ecb-expand-directory-nodes' which does the same for the}
\label{sec-25.5.1}

    directories tree-buffer as the new command `ecb-expand-methods-nodes' for
    the methods tree-buffer (Klaus).

\subsubsection{Rewritten cache-mechanism for directories and sources:}
\label{sec-25.5.2}

    Depending on the value of the option ecb-cache-directory-contents the
    following is cached for a directory:
\begin{itemize}
\item The disk-contents of the directory means subdirs and files
\item If ecb-show-sources-in-directories-buffer is nil (i.e. sources are
      displayed in the extra ECB-sources-window) then also the complete
      sources-buffer is cached. This results in a speed-boost for big-size
      directories.
\end{itemize}
    The cache of a directory is refreshed and actualized with a POWER-click onto
    the related directory in the directories-buffer (s.b.) (Klaus)

\subsection{Enhancements to the sources tree-buffer:}
\label{sec-25.6}

   
\subsubsection{The option ecb-source-file-regexps now works on a directory-base, which means}
\label{sec-25.6.1}

    for each directory-regexp the files to display can be specified. Default is
    an all-matching directory-regexp. Old saved values are autom. upgraded
    (Klaus).

\subsection{Enhancements to the layout-customization:}
\label{sec-25.7}


\subsubsection{Naming and managing of layouts has been changed! Now a layout is not longer}
\label{sec-25.7.1}

    identified by an integer but by an arbitrary string! Example: The layout
    with index 0 in ECB <= 1.80 is now named ``left1'' in ECB 1.90.

\begin{itemize}

\item Therefore the name of the option 'ecb-layout-nr' has changed to\\
\label{sec-25.7.1.1}

     `ecb-layout-name'! See the docstring of `ecb-layout-name' for the names
     of all built-in layouts.


\item In this context also the type of ecb-show-sources-in-directories-buffer has\\
\label{sec-25.7.1.2}

     changed so it is now a list of layout-names where the sources should be
     displayed in the directories window (or `always/'never). Now every change of
     the layout also checks the value of ecb-show-sources-in-directories-buffer.
  

\item This change also introduces three new commands:\\
\label{sec-25.7.1.3}

\begin{itemize}
\item ecb-change-layout: Change layouts by selecting a name
\item ecb-delete-new-layout: Delete an user-created (by `ecb-create-new-layout')
       layout.
\item ecb-show-layout-help: Displays the docstring of a layout which at least
       for built-in layouts contains a picture of the outline of this layout.
\end{itemize}
     All three commands offer TAB-completion for easy selecting a layout name.


\item ECB autom. upgrades the following options to theirs new names/types:\\
\label{sec-25.7.1.4}

\begin{itemize}
\item ecb-layout-nr: The stored value is transformed to the new name of the
       related layout and stored in `ecb-layout-name'
\item ecb-toggle-layout-sequence: The stored value is transformed to the new
       type with layout-names instead of layout-numbers.
\item ecb-major-modes-activate: The stored value is transformed to the new type
       with layout-names instead of layout-numbers.
\item ecb-layout-window-sizes: The stored value is transformed to the new type
       with layout-names instead of layout-numbers.
\item ecb-show-sources-in-directories-buffer: Reset to the new default value.
\end{itemize}
     (Klaus)

\end{itemize} % ends low level
\subsubsection{Adding a new layout type "left-right" which allows the ECB-tree-windows to}
\label{sec-25.7.2}

    be located at the left and the right side of the ECB-frame. See the new
    layouts ``leftright1'', ``leftright2'' and ``leftright3'' for examples. Other
    layouts of this type can be created or programmed very easy with
    `ecb-create-newlayout' rsp. the macro `ecb-layout-define'. (Klaus)

\subsubsection{Now ECB offers a command `ecb-create-new-layout' for interactively creating}
\label{sec-25.7.3}

    new layouts ``by example''. Read the online-help. (Klaus)

\subsubsection{Rewritten the mechanism for storing customized window-sizes. See option}
\label{sec-25.7.4}

    ecb-layout-window-sizes and the command ecb-store-window-sizes and
    ecb-restore-window-sizes. Now the sizes are always saved as fractions of
    the width (rsp. height) of the ECB-frame, so the stored sizes are always
    correct regardless of the current frame-size (Klaus). Suggested and first
    implementation by Geert Ribbers [geert.ribbers@realworld.nl]

\subsection{General Enhancements of the tree-windows:}
\label{sec-25.8}


\subsubsection{Introduced a POWER-click (was SHIFT-click before) in the ECB-tree-windows -}
\label{sec-25.8.1}

    see documentation of ecb-primary-secondary-mouse-buttons and see also the
    online-help section ``Usage of ECB --> Using the mouse'' (Klaus).

\subsubsection{Now the ECB-tree-buffers offer in their popup-menus a "Grep" and a "Grep}
\label{sec-25.8.2}

    recursive'' (= grep-find) command where you can perform an easy grep in every
    directory without buffer-switching before. See also the new options
    `ecb-grep-function' and `ecb-grep-find-function'. (Klaus)

\subsubsection{Modified syntax-table for all tree-buffers with all paren, braces and}
\label{sec-25.8.3}

    brackets handled as whitespace so for example no paren-matching takes place
    (Klaus).
    
\subsection{Enhancements to the layout-engine of ECB:}
\label{sec-25.9}

   
\subsubsection{Added a complete new section "The layout-engine of ECB" to the online-help}
\label{sec-25.9.1}

    which describes in detail how to program new layouts and new special
    windows. This is for foreign packages which want to display own informations
    in own special windows and synchronizing these windows with the edit-window
    of ECB (a graphical debugger could be an example). Also a full working
    example is added in the new file ecb-examples.el. (Klaus)

\subsubsection{Completely rewritten how to program new layouts. Now there is a new macro}
\label{sec-25.9.2}

    `ecb-layout-define' which offers very easy programming of new layouts.
    (Klaus).

\subsection{Enhancements to the compile-window}
\label{sec-25.10}


\subsubsection{ECB now has a list of valid compilation buffers within the ECB menu. (Kevin)}
\label{sec-25.10.1}


\subsubsection{ECB now supports `ecb-enlarged-compilation-window-max-height' which can be}
\label{sec-25.10.2}

    used for defining the max. height of the ecb-compile-window if enlarged by
    `ecb-toggle-enlarged-compilation-window'. (burton)

\subsubsection{Added new command 'ecb-toggle-compile-window': Toggle the visibility of the}
\label{sec-25.10.3}

    compile-window of ECB (Klaus).

\subsection{Speedbar is now integrated in ECB and can be used instead of the standard}
\label{sec-25.11}

   ECB-directories buffer (or instead of all tree-buffers if you like :-). Most
   of the important basic work has been done by Kevin A. Burton.
    
\subsubsection{See new option 'ecb-use-speedbar-for-directories' and the node "Integrating}
\label{sec-25.11.1}

    speedbar'' in the online-help. (Klaus)

\subsubsection{Version-checking for correct speedbar version (>= 0.14beta1). For lower}
\label{sec-25.11.2}

    versions ecb-use-speedbar-for-directories has no effect. (Klaus)

\subsubsection{ecb-speedbar and speedbar first loaded direct before drawing the layout}
\label{sec-25.11.3}

    with integrated speedbar. (Klaus)

\subsubsection{The speedbar-commands speedbar, speedbar-get-focus are adviced during}
\label{sec-25.11.4}

    running speedbar within ECB, so they behave in a senseful way. (Klaus)

\subsection{Enhancements to the ehell-support of ECB:}
\label{sec-25.12}

   
\subsubsection{ecb-eshell now recenters after command execution. (burton)}
\label{sec-25.12.1}


\subsubsection{ecb-eshell now recenters if there are any window resizes.  This is done so}
\label{sec-25.12.2}

    that the prompt is always at the bottom of the window. (burton)

\subsubsection{ecb-eshell integration now protects against buffer-list modifications during}
\label{sec-25.12.3}

    transparent eshell updates in the compile window.  In the past, if the eshell
    was running, and you want from buffer ``foo'', to ``bar'', and then tries to go
    back to ``foo'' the next entry in your buffer list is ``*eshell*'' so this would
    break a lot of completion tools like `switch-to-buffer.  (burton)

\subsubsection{Added online documentation for the eshell integration. See section}
\label{sec-25.12.4}

    ``Tips and tricks''. (Klaus)

\subsection{New hooks:}
\label{sec-25.13}

  
\subsubsection{New or renamed hooks:}
\label{sec-25.13.1}

\begin{itemize}
\item ecb-redraw-layout-after-hook (new)
\item ecb-redraw-layout-before-hook (new)
\item ecb-hide-ecb-windows-after-hook (new)
\item ecb-hide-ecb-windows-before-hook (was ecb-hide-ecb-windows-hook)
\item ecb-show-ecb-windows-after-hook (new)
\item ecb-show-ecb-windows-before-hook (was ecb-show-ecb-windows-hook)
\end{itemize}
    Read the documentation! (Klaus)

\subsubsection{Added two general hooks:}
\label{sec-25.13.2}

\begin{itemize}
\item ecb-before-activate-hook: Activate ECB only if all added hooks return not
      nil
\item ecb-before-deactivate-hook: Deactivate ECB only if all added hooks return
      not nil
\end{itemize}
    (Klaus)

\subsubsection{Added 5 new hooks for the ECB tree-buffers:}
\label{sec-25.13.3}

\begin{itemize}
\item ecb-common-tree-buffer-after-create-hook
\item ecb-directories-buffer-after-create-hook
\item ecb-sources-buffer-after-create-hook
\item ecb-methods-buffer-after-create-hook
\item ecb-history-buffer-after-create-hook
\end{itemize}
    These hooks are called directly after tree-buffer creation so they can be
    used for example to add personal local key-bindings either to all
    tree-buffers (ecb-common-tree-buffer-after-create-hook) or just to a certain
    tree-buffer. Please read the documentation. (Klaus)



\section{Changes for ECB version 1.80}
\label{sec-26}


\subsection{ECB now requires:}
\label{sec-26.1}

\begin{itemize}
\item Latest stable release of semantic: Version 1.4
\item Latest stable release of eieio: Version 0.17
\end{itemize}
   ECB checks at start-time if the correct versions are available and if not
   ECB offers you to download and install them from within Emacs. After this
   you have only to restart Emacs and you are fine. (Klaus)

\subsection{Fixed bugs:}
\label{sec-26.2}

   
\subsubsection{Fixed: Now no empty tool-tips are displayed if mouse moves over nodes in the}
\label{sec-26.2.1}

    tree-buffers of ECB which can be displayed completely in the tree-window.
    (Klaus)

\subsubsection{Fixed a bug in ecb-eshell.  If we are running with current buffer sync, and}
\label{sec-26.2.2}

    there is garbage on the current line, when we change buffers this garbage will
    be executed which could result in a `command not found' or even worse, a
    damaging command executed.  We now cleanse the command line prior to directory
    changed. (Kevin A. Burton) 

\subsubsection{Fixed a bug in ecb-speedbar.  The speedbar was doing its own updating outside}
\label{sec-26.2.3}

    of the ECB.  This means that if you were to change buffers in a frame outside
    of the ECB frame, the ECB's Speedbar would be updated to reflect a buffer not
    within the ECB.  We now disable Speedbar's automatic update and have ECB
    handle this.  (Kevin A. Burton) 

\subsubsection{Fixed a bug: Sometimes there occurs a a "wrong-type-argument"-error (or}
\label{sec-26.2.4}

    ``destroyed extent''-error with XEmacs) under some circumstances. Now ECB
    auto-reparses the buffer and after that ECB is in correct state again.
    (Klaus)

\subsubsection{Fixed formatting of "Parent" nodes in token buffer. (Jesper)}
\label{sec-26.2.5}


\subsection{Byte compiling ECB either interactive with `ecb-byte-compile' or with the}
\label{sec-26.3}

   supplied Makefile now checks also the versions of semantic and eieio.
   (Klaus)

\subsection{ECB now displays the section "First steps" of the online-manual after}
\label{sec-26.4}

   activating first time. (Klaus)

\subsection{Now the adviced versions of `delete-window' and `delete-other-windows' can}
\label{sec-26.5}

   also handle the optional WINDOW argument of the original versions correctly
   so the adviced versions now also work correct if called from program (Klaus).

\subsection{New adviced window command: `delete-windows-on' now works correct with ECB.}
\label{sec-26.6}

   See `ecb-advice-window-functions'. (Klaus)

\subsection{In the ECB-directories buffer now F4 adds a new source-path instead of F1.}
\label{sec-26.7}

   That is because F1 is such an important key in many OS and tools (e.g. opens
   help in all windows-programs), so ECB should not hardly ``rebind'' it. (Klaus)

\subsection{Enhancing the option 'ecb-window-sync': Now the synchronization can take}
\label{sec-26.8}

   place for all buffers which have a relation to files AND directories, i.e.
   now also for dired-buffers. But per default this is switched off, see the
   doc-string. (Klaus)

\subsection{Now the ECB online help is available in the standard Info-format and also in}
\label{sec-26.9}

   HTML-format. With the new option `ecb-show-help-format' you can choose which
   format is used by the function `ecb-show-help'. (Klaus)

\subsection{New feature and command: ecb-download-ecb. With this function you can}
\label{sec-26.10}

   download and install any arbitrary ECB-version from the ECB-website from
   within Emacs. This is especially useful for upgrading to the latest
   version. (Klaus)

\subsection{New debug mode for ECB which prints some debug informations in the message}
\label{sec-26.11}

   buffer for certain critical operations with semantic-overlays/extends.
   New option `ecb-debug-mode' (also available in the Help-menu of ECB).
   (Klaus)

\subsection{Some default key-bindings have changed; see `ecb-key-map'. (Klaus)}
\label{sec-26.12}


\subsection{New feature: Possibility to define a sequence of layouts with new option}
\label{sec-26.13}

   `ecb-toggle-layout-sequence' and toggle very fast between them via
   `ecb-toggle-layout'. Read the online-help section ``Simulation a speedbar
   without an extra frame''! (Klaus).

\subsection{Two new hooks: ecb-hide-ecb-windows-hook and ecb-show-ecb-windows-hook. Read}
\label{sec-26.14}

   the documentation. (Klaus)
   Suggested by Daniel Hegyi <daniel\_{}hegyi@hotmail.com>

\subsection{Two new options: ecb-major-modes-activate and ecb-major-modes-deactivate.}
\label{sec-26.15}

   Allow (de)activation of ECB on major-mode base. Instead of deactivation also
   just hiding the ECB-windows is possible. Read the docstrings. (Klaus)
   Suggested by Daniel Hegyi <daniel\_{}hegyi@hotmail.com>

\subsection{Redefining of the option 'ecb-compile-window-temporally-enlarge': Now you}
\label{sec-26.16}

   can specify that the compile-window is auto. enlarged after selecting it and
   auto. shrinked after leaving it. Read the docstring! (Klaus)
   Suggested by Daniel Hegyi <daniel\_{}hegyi@hotmail.com>

\subsection{Much better and more powerful auto. upgrading-mechanism of incompatible or}
\label{sec-26.17}

   renamed ECB-options to latest ECB-version. See the option
   ecb-auto-compatibility-check! (Klaus)

\subsection{New option ecb-tree-RET-selects-edit-window: Possibility not to select the}
\label{sec-26.18}

   edit-window of hitting RET in a tree-buffer; see the doc-string. There is
   also a new function ecb-toggle-RET-selects-edit-window which is bound to
   [C-t] in each tree-buffer. (Kevin A. Burton,  Klaus)

\subsection{Better grouping of external methods, i.e. methods which are implemented}
\label{sec-26.19}

   outside the class-definition, like in C++, CLOS and EIEIO. Now this is done
   completely by semantic. (Klaus)

\subsection{For application programmers using the tree-buffer library:}
\label{sec-26.20}

\begin{itemize}
\item The function tree-buffer-create has a new argument NODE-DATA-EQUAL-FN. See
     the doc-string for a description (Klaus).
\item For each tree-buffer a special user-data-storage (for any arbitrary data)
     is created which can be accessed via `tree-buffer-set-data-store' and
     `tree-buffer-get-data-store'. See the docstring of `tree-buffer-create'.
     (Klaus)
\end{itemize}
\section{Changes for ECB version 1.70}
\label{sec-27}


\subsection{New option ecb-bucket-token-display. See doc-string for details. (Klaus)}
\label{sec-27.1}


\subsection{Reorganization of the faces used by ECB. The group 'ecb-face-options'}
\label{sec-27.2}

   contains now all options which type is `face (i.e. the defcustom's) and the
   group `ecb-faces' contains all face-definition (the defface's). In addition
   it is now possible to change easily and fast the general face for ALL
   ECB-tree-buffers (same for the highlighting face). See the docstring of the
   new faces `ecb-default-general-face' and `ecb-default-highlight-face' for
   further details. (Klaus).

\subsection{Smarter incr. search in a tree-buffer with the [end]-key: If there are at}
\label{sec-27.3}

   least two nodes with the same greatest common-prefix than every hit of
   [end] jumps to the next node with this common prefix. (Klaus)

\subsection{New feature ecb-auto-compatibility-check: ECB checks on startup if the}
\label{sec-27.4}

   current value of an ECB-option OPT is not compatible with the type of OPT in
   current ECB-release. This is done during activation. If there are
   incompatible options ECB resets them to the default value of current ECB
   release and displays after activation in a temporary window which options
   have been reset, incl. their old-value (before the reset) and new-value
   (after the reset). (Klaus)

\subsection{New feature ecb-tree-navigation-by-arrow: Smarter navigation in the}
\label{sec-27.5}

   tree-windows with left- and right-arrow key (Klaus).
   Thanks for the suggestion and a first implementation to Geert Ribbers
   <geert.ribbers@realworld.nl>!

\subsection{New feature 'ecb-type-token-display'. See the doc-string of this new option}
\label{sec-27.6}

   for a description. (Klaus)

\subsection{Now the option 'ecb-post-process-semantic-tokenlist' has per default an}
\label{sec-27.7}

   entry for emacs-lisp-mode, so for eieio-code the methods (defmethod) of a
   class are grouped with respect to the class they belong. The new default
   value of this option does this for c++- and eieio-code. (Klaus)

\subsection{Now in the ECB-tree-buffers RETURN works like the primary button (already in}
\label{sec-27.8}

   previous versions) and C-RETURN works like the secondary button (new in
   version 1.70). For an explanation of primary and secondary see the option
   ecb-primary-secondary-mouse-buttons. (Klaus)

\subsection{New feature: Now the token-display function can be defined separated for}
\label{sec-27.9}

   each major-mode. Furthermore the token-display for C++ has enhanced,
   especially for template-classes (but template displaying is first available
   with semantic > version 14beta13!).
   For this the option ecb-token-display-function has been completely rewritten
   and is now incompatible with the old values before ECB 1.61. (Klaus)

\subsection{Fixed a bug with mouse-avoidance-mode (only relevant for GNU Emacs 20.X)}
\label{sec-27.10}

   Klaus. 

\subsection{New feature, ecb-select-edit-window-on-redraw.  Mostly used if you are running}
\label{sec-27.11}

   with ecb-redraw-layout-quickly and want to make sure you are always in the
   edit window. (Kevin)

\subsection{New feature, ecb-auto-activate.  If non-nil we can start the ECB after Emacs}
\label{sec-27.12}

   is started.  See `ecb-auto-activate' for more information.  (Kevin)

\subsection{New package, ecb-cycle.el.  Supports cycling through all known compilation}
\label{sec-27.13}

   buffers. (Kevin)
\begin{itemize}
\item ecb-cycle-switch-to-compilation-buffer - switch to available compilation
       buffers.
\item ecb-cycle-through-compilation-buffers - cycle through all compilation
       buffers.
\end{itemize}
  
\subsection{Much better performance of the directory and file-browser in ECB, especially}
\label{sec-27.14}

   for directories with a lot of entries (means \~{} >1000, dependent on your
   machine and the net-connection in case of network-drives). See also the new
   feature: ecb-cache-directory-contents. (Klaus)

\subsection{ecb-eshell now supports customization.  (Kevin)}
\label{sec-27.15}


\subsection{New feature: ecb-layout-always-operate-in-edit-window. Look at the}
\label{sec-27.16}

   documentation. (Kevin \& Klaus)

\subsection{Fixed a bug in `ecb-toggle-ecb-windows'.  For starters we weren't setting}
\label{sec-27.17}

   `ecb-window-hidden' when we redraw the layout.  This means that the ecb
   windows could be hidden but never shown.
   Other things:
\begin{itemize}
\item ecb-windows-hidden now has documentation
\item we display status messages when we hide windows. (Kevin)
\end{itemize}
\subsection{Fixed bug in `ecb-redraw-layout-quickly'.  If for come reason any of the ECB}
\label{sec-27.18}

   windows are not alive, we do not return with a hard error.  Instead, we use
   the scratch buffer.  (Kevin)

\subsection{New function ecb-toggle-enlarged-compilation-window allows users to expand and}
\label{sec-27.19}

   then contract the compilation window with a new key binding C-c . / (Kevin)

\subsection{new option ecb-add-path-for-not-matching-files. Look at the documentation}
\label{sec-27.20}

   (Klaus).


   
\section{Changes for ECB version 1.60}
\label{sec-28}


\subsection{ecb-highlight-token-header-after-jump is now a boolean; which face is used}
\label{sec-28.1}

   for highlighting is defined in ecb-token-header-face. (Klaus)

\subsection{New option 'ecb-post-process-semantic-tokenlist': Special post-processing of}
\label{sec-28.2}

   the semantic tokenlist for certain major-modes. This is useful for
   c++-mode. Now with the default value of this option all methods in a
   c++-implementation-file (no class/method-declaration but only method
   implementations) are grouped with respect to the class they belong. (Klaus)

\subsection{Now also with C++ Sources the method- and variable-protection is displayed}
\label{sec-28.3}

   correct in the ECB-method buffer (Klaus).

\subsection{Fixed a bug which prevented ECB using the root path "/" on Unix-like systems}
\label{sec-28.4}

   as a source-path. (Klaus)

\subsection{If a file via the standard-mechanisms of Emacs (e.g. find-file) is opened,}
\label{sec-28.5}

   then the auto. window sync (see `ecb-window-sync') has worked only correct
   if the new file is located in any of the paths in `ecb-source-path'. Now ECB
   adds the path of the new file at least temporally via `ecb-add-source-path'
   to the list of paths in `ecb-source-path', so the window-sync works always
   correct. ECB asks the user if the new path should saved also for future
   sessions. (Klaus)

\subsection{Added ecb-redraw-layout-hooks so that code can be executed after we redraw the}
\label{sec-28.6}

   ecb layout.

\subsection{New hook ecb-current-buffer-sync-hook allows developers to add code to be}
\label{sec-28.7}

   evaluated after the ECB is synchronized with the current buffer.

\subsection{Now all faces used by ECB for highlighting and displaying it��s own stuff are}
\label{sec-28.8}

   customizable. See new group `ecb-faces' (Klaus).

\subsection{ECB is now fully prepared for Emacs 21. The new feature}
\label{sec-28.9}

   `ecb-show-node-info-in-minibuffer' is implemented with the new `help-echo
   property of Emacs 21, therefore the ugly `track-mouse' mechanism is not
   needed anymore with Emacs 21. (Klaus).

\subsection{The options `ecb-show-node-name-in-minibuffer',}
\label{sec-28.10}

   `ecb-show-complete-file-name-in-minibuffer' and
   `ecb-show-file-info-in-minibuffer' are gone and have been replaced by one
   single new option `ecb-show-node-info-in-minibuffer' where you can define
   separately for every tree-buffer when and which node info should be
   displayed in the minibuffer (Klaus).

\subsection{ecb-auto-expand-directory-tree now offers two options:}
\label{sec-28.11}

\begin{itemize}
\item best: Expand always the best matching source-path for a file
\item first: Expand always the first matching source-path for a file (Klaus)
\end{itemize}
\subsection{Fixed a bug in auto. expanding directories if a source-path has an alias.}
\label{sec-28.12}

   (Klaus)

\subsection{If mouse is moved over an alias in the directories buffer then the real path}
\label{sec-28.13}

   is shown in the echo-area; see also `ecb-show-node-info-in-minibuffer'
   (Klaus).

\subsection{If mouse is moved over an history-entry then the full path is shown in the}
\label{sec-28.14}

   echo-area so you can distinct better between two entries with the same name
   but with different paths; see also `ecb-show-node-info-in-minibuffer'
   (Klaus).

\subsection{New option for the history: ecb-kill-buffer-clears-history defines if}
\label{sec-28.15}

   `kill-buffer' should also remove the corresponding history entry. There are
   several options (Klaus).


   
\section{Changes for ECB version 1.52}
\label{sec-29}


\subsection{Added EIEIO requirement to ECB.}
\label{sec-29.1}




\section{Changes for ECB version 1.51}
\label{sec-30}


\subsection{Now the names of all ECB buffers begin with a SPC per default. (Klaus)}
\label{sec-30.1}


\subsection{ECB now can handle not only full but also partial buffer reparsing if done}
\label{sec-30.2}

   by semantic and other tools (e.g. JDE >= 2.2.9). The method buffer will
   always be uptodate if for a major-mode auto. reparsing after buffer changes
   is enabled (the auto. buffer reparsing itself is not an ECB feature but must
   be supplied by the major-mode, e.g. JDE). (Klaus)

\subsection{Incremental node-search in the ECB-buffers now uses the value of}
\label{sec-30.3}

   `case-fold-search'. (Klaus)

\subsection{Complete new customization of the ECB-key-bindings. The option}
\label{sec-30.4}

   `ecb-prefix-key' has been removed. There is one new option `ecb-key-map'
   where you can customize all key-settings of ECB (including a common
   prefixkey). (Klaus)

\subsection{ecb-add-source-path and ecb-delete-source-path now ask if saving should be}
\label{sec-30.5}

   done for future sessions (Klaus).

\subsection{Added new navigator functionality. Makes it easy to go to the back and forward}
\label{sec-30.6}

   in navigated tokens and buffers. See ecb-nav-goto-next and 
   ecb-nav-goto-previous. (Jesper)

\subsection{Added an option to narrow the buffer to the token that is jumped to. See}
\label{sec-30.7}

   ecb-token-jump-narrow. (Jesper)

\subsection{Fixed a small bug that placed the window start at the beginning of the token}
\label{sec-30.8}

   instead of the beginning of the token line. (Jesper)


      
\section{Changes for ECB version 1.50}
\label{sec-31}


\subsection{Fixed a minor bug that occured when de-activating ECB under XEmacs. (Jesper)}
\label{sec-31.1}


\subsection{ecb-layout-window-sizes variable is now an association list, which makes it}
\label{sec-31.2}

   easier to add and remove layouts. (Jesper)

\subsection{Added a new layout (nr. 12). Thanks to Nick Cross}
\label{sec-31.3}

   <nick.cross@prismtechnologies.com> for this layout.

\subsection{Each source path can now have an alias that is displayed instead of the path}
\label{sec-31.4}

   name in the directories buffer. (Jesper)

\subsection{The history items can now use the buffer name instead of the file}
\label{sec-31.5}

   name. Customizable with ecb-history-item-name. (Jesper)

\subsection{When jumping to a token the window can be scrolled so that the token starts}
\label{sec-31.6}

   at the top or center of the window. This behavior is customizable with
   the variable ecb-scroll-window-after-jump. (Jesper)

\subsection{Fixed a bug when retrieving parent names for a token. (Jesper)}
\label{sec-31.7}


\subsection{Now the option 'ecb-prefix-key' is customizable so you can define another}
\label{sec-31.8}

   prefix if there are conflicts with other minor-modes or packages (Klaus).


   
\section{Changes for ECB version 1.41}
\label{sec-32}


\subsection{Fixed bug when clicking on token in the methods buffer in XEmacs.}
\label{sec-32.1}




\section{Changes for ECB version 1.40}
\label{sec-33}


\subsection{Tree-incremental-search in the ecb-windows now ignores all non interesting}
\label{sec-33.1}

   stuff:
\begin{itemize}
\item any leading spaces
\item expand/collapse-buttons: [+] rsp. $\boxminus$
\item protection-signs (+, -, \#) in the method-window if uml-notation is used
\item variables types or return-types in front of variable- or method-names.
\item const specifier for variables
\end{itemize}
   This makes incremental-searching in a tree-buffer much faster and easier.
   (Klaus)

\subsection{ECB now uses Semantic DB to find parents of types. (Jesper)}
\label{sec-33.2}


\subsection{Long source paths in the directories buffer are now truncated at the}
\label{sec-33.3}

   beginning. Customizable with variable ecb-truncate-long-names. (Jesper)

\subsection{Tokens can now be sorted by access. (Jesper)}
\label{sec-33.4}


\subsection{Added a function to semantic-clean-token-hooks that just updates the token}
\label{sec-33.5}

   changed instead of re-building the entire token tree. (Jesper)

\subsection{Implemented a token tree cache that stores recently opened buffers' token}
\label{sec-33.6}

   trees. This makes the buffer switching much faster and also saves expanded
   nodes and window positions. The cache is unlimited at the moment. (Jesper)

\subsection{Added a speedbar-like layout nr. 11. This is very useful (like also layout}
\label{sec-33.7}

   nr. 10) either for users with small screens or users which normally do not
   need/want the ECB-windows but sometimes browsing/selecting
   methods/variables. (Klaus)

\subsection{Better extraction of tokens from current buffer. ECB now displays any token}
\label{sec-33.8}

   type in any order. (Jesper)

\subsection{ECB now uses Semantic to display tokens in the methods buffer. (Jesper)}
\label{sec-33.9}


\subsection{Much saver ecb-redraw-layout: Now the layout can be restored ALWAYS}
\label{sec-33.10}

   regardless what messy thing has been done before. (Klaus)

\subsection{New advice for 'other-window-for-scrolling', so all scroll-functions for the}
\label{sec-33.11}

   ``other'' window can also scroll the first edit-window if point stays in the
   second one. (Klaus)

\subsection{ecb-toggle-ecb-windows now preserves the split, the split-amount, the buffer}
\label{sec-33.12}

   contents, the window starts and current point. (Klaus)

\subsection{Some important changes in the ECB-layout concerning displaying compilation-}
\label{sec-33.13}

   and temp-buffers. (Klaus):

\subsubsection{'ecb-select-compile-window' has been gone.}
\label{sec-33.13.1}

  
\subsubsection{'ecb-use-dedicated-windows' is not longer a user-customizable option but}
\label{sec-33.13.2}

    it is always set to t, because this is essential for correct working of
    the ecb-layout engine!

\subsubsection{'ecb-compile-window-temporally-enlarge' is now a boolean option.}
\label{sec-33.13.3}


\subsubsection{Now ECB does all necessary that your edit-window of ECB seems to be a}
\label{sec-33.13.4}

    normal Emacs frame, especially concerning displaying temp-buffers (like
    help- and completion-buffers) and compilation-buffers.
    
\subsubsection{The variables 'compilation-window-height' and 'temp-buffer-shrink-to-fit'}
\label{sec-33.13.5}

    (XEmacs) and `temp-buffer-resize-mode' (GNU Emacs) are now fully supported
    by ECB if no durable compilation-window is shown.

\subsection{New function to toggle visibility of the ECB windows. Now it��s possible to}
\label{sec-33.14}

   hide all ECB windows without deactivating ECB (see `ecb-toggle-ecb-windows')
   (Klaus).

\subsection{ECB is now a global minor mode with it��s own menu "ECB" and it��s own key-map}
\label{sec-33.15}

   with prefix ``C-c .''. New function to (de)activate/toggle ECB with
   `ecb-minor-mode'. (Klaus)

\subsection{Fixed a bug in highlighting current token in the method-buffer when}
\label{sec-33.16}

   font-lock-mode is nil for the source-buffer. (Klaus)

\subsection{New option for highlighting the method-header in the edit-window after}
\label{sec-33.17}

   clicking onto the method in the method-window (like Speedbar does). (Klaus)

\subsection{Function for automatically submitting a problem report to the ECB mailing}
\label{sec-33.18}

   list: ecb-submit-problem-report. (Klaus)

\subsection{Now not only for method-highlighting an idle delay can be set but also for}
\label{sec-33.19}

   synchronizing the ECB windows with current edit window (see option
   ecb-window-sync and ecb-window-sync-delay; default is 0.25 sec delay)
   (Klaus).

\subsection{Smarter highlighting of current method (Klaus).}
\label{sec-33.20}


\subsection{All tree-buffers now have as default-directory the current selected}
\label{sec-33.21}

   directory in the directory buffer. So you can also open files with find-file
   etc. from within the tree-buffers. (Klaus).


   
\section{Changes for ECB version 1.32}
\label{sec-34}


\subsection{Nil parent bug fixed. (Jesper)}
\label{sec-34.1}


\subsection{New advices for find-file and switch-to-buffer (Klaus).}
\label{sec-34.2}


\subsection{Now possible to set an idle delay before the current method is highlighted;}
\label{sec-34.3}

   useful for slow machines but prevents also ``jumping backward/forward'' during
   scrolling within java-classes if point goes out of method-definition into
   class-definition. Default is an idle time of 0.25 seconds. (Klaus).


   
\section{Changes for ECB version 1.31}
\label{sec-35}


\subsection{Parents (extends and implements) of the classes in the current buffer is now}
\label{sec-35.1}

   shown in the methods buffer. (Jesper). Possibility to define a regexp which
   parents should not be shown. (Klaus).

\subsection{Now displaying the complete node name in minibuffer with the track-mouse}
\label{sec-35.2}

   mechanism works also with mouse-avoidance-mode on; the mouse-avoidance will
   be deactivated as long ECB is activated and the node-name display-mechanism
   is on. This refers only to GNU Emacs! (Klaus)

\subsection{Fixed a bug in 'ecb-current-buffer-sync' and 'ecb-redraw-layout' (Klaus)}
\label{sec-35.3}


\subsection{Fixed a bug in loading semantic 1.3.3 or semantic 1.4 (Klaus)}
\label{sec-35.4}




\section{Changes for ECB version 1.30}
\label{sec-36}


\subsection{If not all ECB-tree-windows of current layout are visible at redraw-time}
\label{sec-36.1}

   (`ecb-redraw-layout') then the redraw synchronizes the contents of the
   tree-windows with the source displayed in current edit-window (Klaus).

\subsection{Added a section "Tips and Tricks" to the ECB online-help. (Klaus)}
\label{sec-36.2}


\subsection{Added a new layout Nr. 10 for very small screen resolutions where all}
\label{sec-36.3}

   square-centimeters are needed for the editing itself. This layout only
   displays a method-window and a edit-window. (Klaus).

\subsection{The messages displayed after moving the mouse over a node in a tree-buffer}
\label{sec-36.4}

   do not longer wasting the log, means they will not be logged (Klaus).

\subsection{ECB now also works with semantic >= 1.4 (Klaus).}
\label{sec-36.5}


\subsection{Now the ecb-compile-window-height is also preserved after displaying}
\label{sec-36.6}

   temp-buffers (e.g. help-buffers after C-h f) - if you want this. See the
   documentation of the option `ecb-compile-window-temporally-enlarge'. (Klaus)

\subsection{Added menu items for modification of the source paths in the directories}
\label{sec-36.7}

   buffer. (Jesper)

\subsection{Added buttons to directories buffer. (Jesper)}
\label{sec-36.8}


\subsection{Mouse over files, methods etc. now work even if follow-mouse isn't}
\label{sec-36.9}

   activated (Jesper). ECB adds a more intelligent mouse tracking mechanism, so
   not only the mouse-over-node stuff of ECB works now very well and savely but
   also follow-mouse itself works better and saver as without activated ECB.
   (Klaus)

\subsection{Popup menus now work in XEmacs. (Jesper)}
\label{sec-36.10}


\subsection{You can now specify paths with env-vars like \$HOME in the option}
\label{sec-36.11}

   `ecb-source-path'. (Klaus)

\subsection{The user must now confirm if he tries to delete the ECB frame. If he wants}
\label{sec-36.12}

   to proceed then ECB will be first deactivated before deleting the frame.
   This works for all ways to delete a frame (shortcut, window-manager-button,
   \ldots{}). (Klaus)

\subsection{ECB now optionally create its own frame when activated. See the new option}
\label{sec-36.13}

   `ecb-new-ecb-frame'. (Klaus)

\subsection{Intelligent recentering of tree-buffers now completely implemented without}
\label{sec-36.14}

   the function `recenter'. This means no flickering and flashing the whole
   frame anymore after each `recenter'. Now the ECB display is stable like a
   rock :-) (Klaus)

\subsection{Mouse over directories now shows directory path. (Jesper)}
\label{sec-36.15}


\subsection{Methods and variables with point are now highlighted. (Jesper)}
\label{sec-36.16}


\subsection{ECB windows are now updated when saving new files and deleting files. (Jesper)}
\label{sec-36.17}


\subsection{incremental-search in every tree-buffer for easier and faster selecting}
\label{sec-36.18}

   nodes with the keyboard (see new option `ecb-tree-incremental-search' and
   new function `tree-buffer-incremental-node-search') for more details.
   (Klaus)

\subsection{ecb-redraw-layout now restores both edit window buffers and the edit window}
\label{sec-36.19}

   start. (Jesper)

\subsection{Spelling corrections. Thanks to Colin Marquardt}
\label{sec-36.20}

   <colin.marquardt@usa.alcatel.com> for spotting them.

\subsection{Mode line prefixes can now be set individually for each ECB-buffer. There is}
\label{sec-36.21}

   a default-prefix for each buffer but the user can also define a custom
   prefix or no prefix at all. (Jesper/Klaus)

\subsection{Fixed small bug in directory tree that caused directories to be sorted by}
\label{sec-36.22}

   extension. (Jesper)


   
\section{Changes for ECB version 1.20}
\label{sec-37}


\subsection{Now the ECB-buffers are intelligent in displaying nodes (Klaus):}
\label{sec-37.1}

\begin{itemize}
\item Expandable nodes: (e.g. a directory with it��s subdirectories). An
     ECB-tree-window now tries always to make visible as much as possible of
     the sub-nodes of a node if a node is expanded (e.g. by clicking onto it��s
     expand-symbol).
\item Other nodes: A ECB-tree-window is always best filled, means no empty lines
     if there are enough lines to fill the whole window.
\end{itemize}
\subsection{The source file is only parsed if the ECB methods window is visible}
\label{sec-37.2}

   (Jesper).

\subsection{The methods buffer now show fields of EIEIO classes. Only works with}
\label{sec-37.3}

   semantic 1.4. Thanks to Eric M. Ludlam <eric@siege-engine.com> for
   this patch.

\subsection{Now all adviced functions behave only special in the ECB-frame. In each other}
\label{sec-37.4}

   frame they behave exactly like the not adviced functions. (Klaus)

\subsection{Fix "invalid method-buffer"-bug (Klaus, with a lot of help by Eric Ludlam}
\label{sec-37.5}

   and David Ponce).

\subsection{Source files can now be sorted by name, extension or not sorted at all}
\label{sec-37.6}

   (Jesper).

\subsection{Now you can precisely define by an exclude and include regexp which files}
\label{sec-37.7}

   should be displayed by ECB. (Jesper).

\subsection{Jumping to a method/variable from the ECB-method buffer now sets the mark so}
\label{sec-37.8}

   the user can easily jump back (see `ecb-method-jump-sets-mark'). (Klaus)

\subsection{Now you can define a primary and a secondary mouse-button for ECB. See the}
\label{sec-37.9}

   new variable `ecb-primary-secondary-mouse-buttons'.
   Note: The name of the option `ecb-left-mouse-jump-destination' has been
   changed to `ecb-primary-mouse-jump-destination'!

\subsection{Now ECB is really ediff-friendly: During ediff is active the advices of ECB}
\label{sec-37.10}

   are temporally deactivated! (Klaus)

\subsection{Now all tree-buffers (ECB-buffers) are read-only (Klaus)!}
\label{sec-37.11}


\subsection{Better implementation of advising the functions. Now all advice-able}
\label{sec-37.12}

   functions are completely independent from each other (Klaus)!

\subsection{Removes the redundant modeline info and shows important}
\label{sec-37.13}

   information.

\subsection{Added new option `ecb-advice-window-functions'. Enabling the intelligent}
\label{sec-37.14}

   window functions via customize. No hook hacking necessary anymore!
   Now the intelligent window functions are implemented by advicing (Klaus).

\subsection{Better online help display (Klaus).}
\label{sec-37.15}


\subsection{Added new option 'ecb-primary-mouse-jump-destination' (thanks to David Hay}
\label{sec-37.16}

   for suggestion <David.Hay@requisite.com>.

\subsection{Adviced 'other-window' acts now exactly like the original but always related}
\label{sec-37.17}

   to `ecb-other-window-jump-behavior' (Klaus)!

\subsection{Fixed bug with 'ecb-compile-window-height'. Now ECB has always the correct}
\label{sec-37.18}

   compile-window height after compilation-output (compile, grep etc.)

\subsection{New option 'ecb-compile-window-temporally-enlarge'. Allowing}
\label{sec-37.19}

   Emacs-compilation to enlarge the ECB-compile-window temporally.


   
\section{Changes for ECB version 1.10}
\label{sec-38}


\subsection{Automatic expansion of the directory tree.}
\label{sec-38.1}


\subsection{Method parsing is now recursive, so that for example inner classes}
\label{sec-38.2}

   in Java are shown in the methods buffer.

\subsection{Variables are now shown in the methods buffer.}
\label{sec-38.3}


\subsection{Selected files and directories now use the secondary-selection}
\label{sec-38.4}

   face. Thanks to Kevin A. Burton <burton@relativity.yi.org> for this
   patch.

\subsection{Fixed some issues with XEmacs. Thanks to Colin Marquardt}
\label{sec-38.5}

   <colin.marquardt@usa.alcatel.com> for this.

\subsection{Fixed bug with truncating lines in tree-buffer.el.}
\label{sec-38.6}


\subsection{The methods buffer can now be automatically updated when the source}
\label{sec-38.7}

   file is saved.

\subsection{Help text is available with "ecb-show-help".}
\label{sec-38.8}


\subsection{Moving the mouse over a node shows the name in the}
\label{sec-38.9}

   mini-buffer. Currently this only works when follow-mouse is
   activated.


   
\section{Changes for ECB version 1.0}
\label{sec-39}


\subsection{Changed name from JCB to ECB and prefix from "jde-jcb-" to "ecb-".}
\label{sec-39.1}


\subsection{Works with semantic 1.3.1 and higher.}
\label{sec-39.2}


\subsection{Fixed a couple of bugs when creating and deleting files.}
\label{sec-39.3}


\subsection{Grouped customization variables.}
\label{sec-39.4}


\subsection{Contributed by Klaus Berndl <klaus.berndl@sdm.de>:}
\label{sec-39.5}


\subsubsection{A lot of intelligent ECB window-functions (e.g. ecb-delete-window) which}
\label{sec-39.5.1}

    gives you the feeling as if the edit-window is the only window of the ECB
    frame.

\subsubsection{New layout functionality.}
\label{sec-39.5.2}


\subsubsection{The history buffer can optionally be sorted and also cleared.}
\label{sec-39.5.3}


\subsubsection{The package expanded symbol can optionally be placed before the package name.}
\label{sec-39.5.4}


\subsubsection{Syntax highlighting of the methods buffer.}
\label{sec-39.5.5}


\subsubsection{Show argument names and return type in the methods buffer.}
\label{sec-39.5.6}


\subsubsection{History clear function: ecb-clear-history.}
\label{sec-39.5.7}


\subsubsection{Added pop-up menu to the history buffer.}
\label{sec-39.5.8}


\subsubsection{Fixed edit buffer sync bug.}
\label{sec-39.5.9}




\section{Changes for ECB version 0.04}
\label{sec-40}


\subsection{Keyboard navigation in JCB buffers. Use Return to select and Tab to expand.}
\label{sec-40.1}


\subsection{Tree view of packages.}
\label{sec-40.2}


\subsection{Class files can be viewed in the packages buffer.}
\label{sec-40.3}


\subsection{Better handling of mouse events.}
\label{sec-40.4}


\subsection{Automatic update of the packages buffer.}
\label{sec-40.5}


\subsection{Added pop-up menus to the packages and classes buffers. They are activated}
\label{sec-40.6}

   with mouse button 3 (right button).

\subsection{Added method sorting. Thanks to Brian Anderson <brian.anderson@luna.com>}
\label{sec-40.7}

   for this improvement.


   
\section{Changes for ECB version 0.03}
\label{sec-41}


\subsection{Fixed naming of variables.}
\label{sec-41.1}


\subsection{Fixed license and copyright comments.}
\label{sec-41.2}


\subsection{Fixed class file regular expression bug.}
\label{sec-41.3}

   Thanks to Jim Loverde <loverde@str.com> for finding it.

\subsection{Added three new layouts.}
\label{sec-41.4}


\subsection{The JCB buffers can now be (almost) synchronized with the edit window.}
\label{sec-41.5}


\subsection{When clicking on a class or method using mouse button 2, it will be loaded}
\label{sec-41.6}

   into another window. A new customization variable was added for this.

\subsection{Minor changes. Thanks to Jim for his suggestions.}
\label{sec-41.7}


\subsection{Removed the "jde-jcb-window-skips-from-methods-buffer" and}
\label{sec-41.8}

   ``jde-jcb-window-skips-from-history-buffer'' customization variables.

   
   
\section{Changes for ECB version 0.02}
\label{sec-42}


\subsection{Changed prefix from "jcb-" to "jde-jcb-". This will cause JCB customization}
\label{sec-42.1}

   variables to be saved in JDE project files.

\subsection{Lists and parses non-Java file types. Currently the C and Lisp parsers from}
\label{sec-42.2}

   semantic-ex.el are used.

\subsection{History buffer implemented.}
\label{sec-42.3}


\subsection{Layout function implement. Thanks to Jim Loverde <loverde@str.com> for}
\label{sec-42.4}

   helping out with this.

\subsection{Packages matching a regular expression can be excluded from the packages}
\label{sec-42.5}

   buffer. Thanks to Mark Gibson <mark@markg.co.uk> for this feature.


   
\section{Changes for ECB version 0.01}
\label{sec-43}


\subsection{Initial version.}
\label{sec-43.1}




Local variables:
mode: outline
paragraph-separate: ``[  ]*\$''
end:

\end{document}